\chapter{Modello di Saint Venant}
	Il modello più semplice per studiare una trave è formulato da Saint Venant che, tramite delle opportune ipotesi semplificative, permette di facilitare la descrizione del problema.
	
	\figura{7}{0.7}{schemasaintvenant}{schema di riferimento della trave di Saint Venant}{schemasaintvenant}
	
	\begin{concetto}
		Il \textbf{solido di Saint Venant}, figura \ref{schemasaintvenant}, è descritto da una trave di lunghezza $L$ sull'asse congiungente gli estremi $A$ e $B$. La terna di riferimento per l'analisi è posta con $z$ coincidente all'asse della trave, mentre gli assi $x,y$ sono posti in modo da formare un \textbf{sistema baricentrico principale} (per cui $I_{xy} = S_x = S_y = 0$).
		
		Le \textbf{ipotesi semplificative} necessarie per utilizzare il modello sono:
		\begin{itemize}
			\item che la lunghezza $L$ dell'asse sia di molto maggiore delle lunghezze caratteristiche delle sezioni. La trave deve essere rettilinea (o con raggio di curvatura molto alto) e la sezione deve essere costante (o variare blandamente lungo l'asse) e sempre perpendicolare all'asse in ogni punto della trave;
			\item che il mantello $\Gamma$ sia scarico e che non siano presenti azioni di volume $\ov b$. L'unico carico esterno che può essere applicato deve essere posto in corrispondenza delle basi $A$ e $B$.
		\end{itemize}
	\end{concetto}

	In generale nota la distribuzione di azioni superficiali $\ov f_A$ su una base ($A$ in questo caso), è possibile creare un sistema staticamente equivalente di risultante $\ov R_A$ e momento $\ov M_A$ rispetto al sistema baricentrico della trave:
	\[ \ov R_A= \int_{\Omega_A} \ov f_A\, d\Omega \qquad \ov M_A = \int_{\Omega_A} \ov r \times \ov f_A\, d\Omega \]
	
	In condizioni statiche è inoltre necessario che la risultante delle forze sul corpo sia nulla, da cui l'equazione $\ov R_A + \ov R_B=  \ov 0$, mentre bilanciando il momento rispetto al polo $A$ si ricava che $\ov M_A + \ov M_B + \ov O_A\ov O_B \times \ov R_B = \ov 0$.
	
	\begin{concetto}
		Si definisce dunque il \textbf{postulato di Saint Venant} per il quale \textit{noto che sulle basi $A,B$ della trave la tensione interna coincide con quella applicata dalle forze distribuite $\ov f_A,\ov f_B$, ad una distanza paragonabile alla dimensione caratteristica massima della base lo stato di tensione non dipende più dalla distribuzione delle azioni $\ov f$, ma solamente dalla risultante delle forze $\ov R$ e dal momento risultante $\ov M$.} \\
		La distanza dalla base rispetto alla quale lo stato di tensione dipende solamente dalla risultante è chiamata \textbf{distanza di estinzione}; nella fascia compresa tra base e distanza di estinzione non vale dunque il modello di Saint Venant.
	\end{concetto}
	\begin{osservazione}
		La distanza di estinzione è tanto minore quanto più compatta è la sezione della trave stessa.
	\end{osservazione}

\section{Stato tensionale}
	Nel modello di Saint Venant è possibile considerare la trave come composta da delle fibre longitutinali (lungo l'asse $z$) che possono reagire solamente a sforzi normali $\sigma_{zz}$ e ad azioni tangenziali del tipo $\tau_{xz},\tau_{yz}$; le altre azioni infatti sarebbero attivate solamente in corrispondenza di azioni sul mantello, cosa che per ipotesi del modello è scaratata. 
	
	In generale il modello può essere accettato fintanto che $\sxx,\syy,\txy \ll \szz,\txz\tyz$ e dunque il \textbf{tensore di Cauchy} si riduce ad uno stato tensionale piano del tipo
	\begin{equation}
		\S = \begin{bmatrix} \label{eq:sv:tensorecauchy}
			0 & 0 & \txz \\ 0 & 0 & \tyz \\ \txz & \tyz & \szz
		\end{bmatrix} \qquad \Rightarrow \quad \ov \szz = \szz \vers k \quad \ov \tau = \tyz \vers i + \txz \vers j
	\end{equation}
	Come è possibile osservare ogni vettore di tensione può essere scomposto in una componente normale alle sezioni $\ov \szz$ e in una componente di tipo tangenziale $\ov \tau$.
	
	\begin{concetto}
		La \textbf{soluzione} del problema di Saint Venant si basa dunque sul determinare il \textbf{campo scalare} associato all'azione normale $\sigma_{zz}(x,y)$ e al \textbf{campo vettoriale} $\ov \tau(x,y)$ associato alle componenti tangenziali.
	\end{concetto}

	Per il postulato di Saint Venant è possibile osservare che le componenti da determinzare $\szz,\txz,\tyz$ (funzioni della posizione), oltre la distanza di estinzione, sono indipendente della azioni di superficie $\ov f$ ma dipendono solamente dalle risultanti delle forze $\ov R = \big(T_x,T_y,N\big)$ e dei momenti $\ov M = \big(M_x, M_y, M_z\big)$.
	
	\begin{concetto}
		Nelle fasce di validità del modello di Saint Venant è dunque possibile istituire un'\textbf{equivalenza statica} tra le azioni staticamente equivalenti $\ov R,\ov M$ e lo stato di tensione interno e dunque:
		\begin{equation} \label{eq:sv:equivstatica}
		\begin{split}
			T_x = \int_{\Omega_s} \txz(x,y)\, d\Omega_s  \qquad & T_y =  \int_{\Omega_s} \tyz(x,y)\, d\Omega_s \\
			N = \int_{\Omega_s} \szz(x,y)\, d\Omega_s \qquad & M_z = \int_{\Omega_s} \Big(x\, \tyz(x,y) - y\,\txz(x,y)\Big) \, d\Omega_s \\
			M_x = \int_{\Omega_s}y \, \szz(x,y)\, d\Omega_s \qquad & M_y = \int_{\Omega_s}x \, \szz(x,y)\, d\Omega_s \\
		\end{split}
		\end{equation}
	\end{concetto}
	\begin{osservazione}
		Osservando l'equivalenza statica nel modello di Saint Venant (eq. \ref{eq:sv:equivstatica}) è possibile osservarae come la reazione normale $N$ e i momenti flettenti $M_x,M_y$ siano associati solamente al campo scalare $\szz$, e dunque ad un tensore del tipo
		\[\S = \begin{bmatrix}
			0 & 0 & 0 \\ 0 & 0 & 0 \\ 0 & 0 & \szz
		\end{bmatrix}\]
		
		Al contrario il momento torcente $M_z$ è associato alla presenza delle componenti $\txz$ e $\tyz$ che, di conseguenza, generano delle azioni taglianti $T_x,T_y$; queste azioni di fatto determinano il tensore di Cauchy del tipo
		\[\S = \begin{bmatrix}
		0 & 0 & \txz \\ 0 & 0 & \tyz \\ \txz & \tyz & 0
		\end{bmatrix}\]
		
	\end{osservazione}
	
	\subsection{Equilibrio indefinito, condizioni al contorno e legame costitutivo}
		\paragraph{Equilibrio indefinito} Come visto nell'equazione \ref{eq:sv:tensorecauchy}, lo stato tensionale di una trave sotto le ipotesi di Saint Venant è piano; considerando che per ipotesi le azioni di volume sono nulle ($\ov b=\ov0$) allora è possibile riscrivere le \textbf{equazioni di equilibrio indefinito} per il problema come:
		\[ \pd \txz z = 0 \qquad \pd \tyz z = 0 \qquad \pd \txz x + \pd \tyz y + \pd \szz z = 0 \]
		Le prime due equazioni dimostrano dunque che le \textbf{componenti tangenziali} $\tau_{xz},\tyz$ del vettore $\ov \tau$ sono \textbf{indipendenti} dalla coordinata assiale $z$, mentre la terza equazione lega la divergenza del campo delle azioni di taglio con la variazione lungo l'asse si $\szz$:
		\[ \textrm{div} \big(\ov \tau \big) = - \pd \szz z \begin{cases}
			= 0 \qquad & \textrm{: campo solenoidale, linee di flusso chiuse} \\
			\neq 0 \qquad & \textrm{: campo non solenoidale, linee di flusso aperte} \\
		\end{cases} \]
	
		\paragraph{Condizioni al contorno} L'ipotesi di Saint Venant afferma che il mantello sia sempre scarico; scelto un qualsiasi punto sull contorno avente direzione normale uscente $\vers n_\Gamma = \big(\alpha_x,\alpha_y,0\big)$ al punto considerato sul mantello $\Gamma$, allora si deve verificare che lo stato di tensione in quel punto sia nullo:
		\[  \begin{bmatrix} 
			0 & 0 & \txz \\ 0 & 0 & \tyz \\ \txz & \tyz & \szz
		\end{bmatrix} \begin{pmatrix}
			\alpha_x \\ \alpha_y \\ 0 
		\end{pmatrix} = \begin{pmatrix}
			0 \\ 0 \\ 0
		\end{pmatrix} \qquad \Rightarrow\quad \txz \alpha_x + \tyz \alpha_y = \ov \tau \cdot \vers n_\Gamma = 0 \quad \Leftrightarrow \quad \ov \tau \perp \vers n_\Gamma \]
		Si osserva dunque che sul mantello il vettore delle tensioni tangenziali $\ov \tau$ deve necessariamente essere ortogonale alla normale del mantello $\vers n_\Gamma$ (e dunque deve essere tangente alla superficie del mantello $\Gamma$ stesso).
		
		\begin{nota}
			In questo caso il versore $\vers n_\Gamma$ presenta l'ultima componente nulla in quanto, per ipotesi, la sezione è costante lungo l'asse della trave, e dunque il mantello non può avere componente normale nell'asse $z$. In caso di variazioni blande della sezione si avrebbe l'introduzione di tale fattore, che tuttavia sarebbe approssimabile al valore nullo.
			
		\end{nota}
		
		\paragraph{Legame costitutivo} Considerando che lo stato di tensione della trave è piano con vettore di tensione $\ov \sigma= (0,0,\szz,\tyz,\txz,0)$, allora considerando la matrice di cedevolezza $\C$ del legame costitutivo elastico lineare per il quale
		\[ \exx = - \frac \nu E \szz \quad \eyy -\frac \nu E \szz \quad \varepsilon_{zz} = \frac \szz E \quad \gamma_{xz} = \frac \txz G \quad \gamma_{yz} = \frac \tyz G \quad \gamma_{xy} = 0  \]
		E' inoltre possibile ricavare la formulazione del potenziale elastico (e relativo complementare) $\phi$ e $\psi$ considerando le componenti di tensione presenti nella trave, e dunque
		\begin{align*}
			\phi = \psi & = \frac{\szz^2}{2E} + \frac{\txz^2}{2G} + \frac{\tyz^2}{2G} = \frac{\szz^2}{2E} + \frac{\tau^2}{2G} \\ 
			& = \frac E 2 \varepsilon_{zz}^2 + \frac G 2 \big(\gamma_{xz}^2 + \gamma_{yz}^2\big)
		\end{align*}
		
	\subsection{Estensione al modello di Saint Venant}
		Per quanto riguarda le \textbf{condizioni sull'asse} è possibile accettare come travi rettilinee tutte le trave il cui raggio di curvatura sia molto maggiore delle dimensioni caratteristiche della sezione; per quanto riguarda la \textbf{sezione} inoltre è possibile ammettere tutte le travi le cui sezioni variano molto blandamente (\textit{pendenza} locale del mantello di qualche grado). \\
		Nel caso in cui si abbiano brusche distorsioni dell'asse e delle sezioni è necessario utilizzare dei modelli più raffinati (o integrare dei coefficienti correttivi) e considerare una nuova distanza di estinzione rispetto all'intorno rispetto al quale il modello non è accettabile.
		
		\vspace{3mm}
		Per quanto riguarda le \textbf{condizioni di carico} è possibile considerare il modello come valido, in presenza di carichi distribuiti sul mantello, ogni qualvolta che 
		\[ \sxx,\syy,\txy \ll \szz,\txz\tyz  \]
		In questa situazione gli effetti di carico sul mantello vengono trascurati in quanto irrisori rispetto alle altre azioni; in caso di carichi concentrati, come nel caso della geometria variabile, è necessario utilizzare modelli più raffinati o coefficienti correttivi introducendo un'ulteriore distanza di estinzione.
		
\section{Soluzione generale nelle tensioni}
	Per risolvere il problema generale delle tensioni della trave secondo il modello di Saint Venant si utilizza un approccio semi-inverso per il quale si suppone una soluzione del problema e si verifica che la stessa sia coerente con le equazioni caratteristiche.
	
	\begin{concetto}
		In primo luogo è possibile studiare il campo scalare $\szz(x,y)$ associato all'azione normale $N$ e ai momenti flettenti $M_x,M_y$; in assenza di momento torcente $M_z$, il campo vettoriale $\ov \tau$ sarebbe nullo solamente se i momenti flettenti fossero costanti lungo il solido, in quanto
		\[ T_y = \frac{dM_x}{dz}=0 \qquad T_x = - \frac{dM_y}{dz} = 0 \]
	\end{concetto}		
	
	\paragraph{Ipotesi deformativa} Osservando la deformazione del concio elementare che costituisce la trave si è osservato che la sezione della trave risulta sempre essere perpendicolare all'asse della trave stesso. Per le considerazioni appena mostrate, le azioni di momento flettente sono supposte costanti lungo tutta la trave e dunque rappresentano un carico simmetrico per ogni concio elementare.
	
	\figura{7}{0.7}{deformazioneconcio}{deformazione del concio elementare soggetto a solo forza normale (sinistra) e solo momento flettente (destra).}{deformazioneconcio}
	
	\figura{7}{0.7}{assedeformato}{lungo l'asse della trave ogni sezione deve essere sempre perpendicolare all'asse stesso.}{assedeformato}
		
	\subsection{Campo di deformazione, di tensione ed energia potenziale}
		
		Per quanto riguarda il \textbf{campo di deformazione}, come si osserva in figura \ref{assedeformato}, si vuol far si che le sezioni siano sempre perpendicolari all'asse della trave.
		\begin{concetto}
			Questo ci permette di ipotizzare che il \textbf{campo di deformazione} $\varepsilon_{zz}$ sia costante o quanto meno lineare lungo gli assi $x,y$, assumendo dunque una formulazione del tipo
			\begin{equation} \label{eq:sv:ipotesideflin}
				\varepsilon_{zz} = ax + by + c
			\end{equation}
		\end{concetto}
		In questo modo è possibile ricondurre la sezione deformata all'equazione di un piano in coordinate $x,y$. Dalle ipotesi di simmetria del cacirco sul concio elementare è possibile asserire le seguenti relazioni per quanto riguarda le deformazioni angolari:
		\[ \gamma_{xz} = \gamma_{yz} = 0  \]
		
		Considerando invece lo \textbf{stato di tensione}, esso può essere analizzato con la legge di Hooke per la quale la tensione $\szz$ è legata alla deformazione $\varepsilon_{zz}$ in funzione del modulo di Young $E$ secondo la legge $ \szz = E\varepsilon_{zz} = E\big(ax + by + c\big)$; unendo a questo risultato le prime due equazioni di equilibrio si ottiene al deformazione lungo gli assi $x$ e $y$ come:
		\[ \Rightarrow \quad \exx = \eyy = - \frac \nu E \szz = - \nu \varepsilon_{zz} \]
		
		\begin{nota}
			In questo caso si sta discutendo il caso in cui si ha la presenza del solo campo scalare $\szz$, e non del campo vettoriale $\ov \tau$.
		\end{nota}
		Per ipotesi la tensione della trave $\szz$ deve essere costante lungo la sezione, e dunque la terza equazione di equilibrio si riduce all'espressione
		\[ \pd \szz z = 0 \]
		
		\paragraph{Equivalenza statica} Per ogni sezione di concio elementare nel modello di Saint Venant deve dunque insistere una relazione di equivalenza statica tra le azioni interne e lo stato di tensione del materiale; come ipotesi iniziale dell'analisi dello stato di tensione normale si erano considerati nulli i termini $T_x,T_y,M_z$ in quanto indipendenti da $\szz$. Al contrario è invece possibile stabilire l'equivalenza statica per l'azione normale e i momenti flettenti dipendenti dal campo scalare $\szz$ come segue:
		\begin{align*}
			N & = \int_A \sigma_{zz} \, dA = \int_A Eax\, dA + \int_A Eby \, dA + \int_A Ec\, dA \\
			& = \cancel{EaS_y} + \cancel{EbS_x} + EcA  \\
			M_x & \int_A \sigma_{zz} y \, dA = \int_A Eaxy\, dA + \int_A Eby^2 \, dA + \int_A Ecy\, dA \\
			& = \cancel{EaI_{xy}} + EbI_{xx}+ \cancel{EcS_x}  \\
			M_y & -\int_A \sigma_{zz} x \, dA = -\int_A Eax^2\, dA - \int_A Ebxy \, dA - \int_A Ecx\, dA \\
			& = -EaI_{yy} -\cancel{ EbI_{xy}}- \cancel{EcS_y}  
		\end{align*}
		\begin{nota}
			I termini $S_y,S_x,I_{xy}$ possono essere trascurati in quanto, come ipotesi iniziale, si era considerato il sistema di riferimento come baricentrico, e dunque tali contributi sono conseguentemente sempre nulli.
		\end{nota}
		
		\begin{concetto}
			Dalle relazioni appena riportate è dunque possibile stabilire analiticamente i coefficienti di linearità $a,b,c$ presenti in equazione \ref{eq:sv:ipotesideflin}; in particolare è possibile definire il \textbf{campo di tensione} e di \textbf{deformazione} nel caso di sola componente scalare $\szz$ di tensione come
			\begin{equation} \label{eq:sv:szznormale}
				\szz = \frac NA + \frac \Mx \Ixx y - \frac \My \Iyy x  \qquad \varepsilon_{zz} = \frac N{EA} + \frac \Mx {E\Ixx} y - \frac \My {E \Iyy} x
			\end{equation}
		\end{concetto}
	
		\paragraph{Energia di deformazione} A questo punto è anche possibile calcolare l'\textbf{energia di deformazione elastica} $U_e$ per un materiale elastico isotropo lineare; per questo materiale infatti è noto il  potenziale elastico $\phi$ e relativo complementare $\psi$ che, integrati sul volume, determinano l'energia elastica:
		\[ \phi = \psi = \frac 1 2 \frac {\szz^2}E = \frac 1 2 E \varepsilon_{zz}^2 \qquad \Rightarrow \quad U_e = \int_V \phi = \int_L \underbrace{\int_A \frac 1 2 \frac {\szz^2}E\, dA }_{dU_e/dz} \, dz \]
		
		Avendo verificato in precedenza che la tensione $\szz$ è indipendente dalla coordinata $z$ è necessario valutare l'integrale associato alla variazione di energia lungo l'asse $dU_e/dz$. Nota l'espressione della tensione (eq. \ref{eq:sv:szznormale}) è dunque possibile valutare tale contributo come
		\begin{equation}
		\begin{split}
			\frac{dU_e}{dz} & = \frac 1{2E} \int_A \left(  \frac NA + \frac \Mx \Ixx y - \frac \My \Iyy x \right)^2\, dA \\
			& \frac 1{2E} \int_A \left(\frac{N^2}{A^2} + \frac{\Mx^2}{\Ixx^2}y^2 + \frac{\My^2}{\Iyy^2}x^2 + \cancel{\textrm{altri termini}} \right)\, dA \\
			& = \frac1 {2E} \left( \frac{N^2}A + \frac{\Mx^2}\Ixx + \frac {\My^2}\Iyy \right)
		\end{split}
		\end{equation}
		\begin{nota}
			Sviluppando opportunamente il quadrato dei contributi di $\szz$ è possibile osservare che gli altri membri, indicati come \textit{altri termini}, si elidono per via delle simmetrie nella scelta del sistema di riferimento baricentrico.
		\end{nota}

\section{Analisi della tensione normale} \label{sec:sv:normale}
	Vigendo il principio della sovrapposizione degli effetti è possibile descrivere lo stato di tensione analizzando l'effetto che ogni azione interna ha sulla trave; partendo analizzando l\textbf{azione normale} $N$, per come visto in equazione \ref{eq:sv:szznormale}, il campo scalare di tensione e conseguentemente di deformazione associato a tale azione è descritto dalle equazioni
	\[ \szz = \frac NA  \qquad \ezz = \frac N {EA} \qquad \exx = \eyy = -\nu \ezz = -\nu \frac N{EA} \]  
	\begin{osservazione}
		I termini $\sxx, \syy, \txy,\txz,\tyz, \gamma_{xy},\gamma_{xz}, \gamma_{yz}$ sono tutti nulli per tutte le ipotesi fin'ora considerate sul modello di Saint Venant.
	\end{osservazione}
	\begin{concetto}		
		Per integrazione delle equazioni di congruenza (\textbf{RIFERIMENTI}) associate al \textbf{campo di spostamento} $\ov u = (u,v,w)$ è possibile determinare le componenti scalari $u,v,w$ in funzione delle coordinate $x,y,z$ dei punti della trave:
		\begin{equation}\label{eq:sv:camponormale}
			\ov u = \left\{ \begin{split}
				u &= -\nu \frac N {EA} x + u_0(y,z) \\
				v &= -\nu \frac N {EA} y + v_0(x,z) \\
				w &= \frac N {EA} z + w_0(x,y) \\
			\end{split} \right.
		\end{equation}
	\end{concetto}
	In questa formulazione i termini $\big(u_0,v_0,w_0\big)$ sono delle funzioni di moto rigido dipendenti solamente dalla scelta del sistema di riferimento per per quali le equazioni di congruenza devono valere $\asmat{\partial} \ov u_0 = \ov 0$.
	\begin{osservazione}
		Essendo il termine $w_0$ costante e dipendente dalla scelta del sistema di riferimento, allora in generale la componente $w$ del campo di spostamento (che è la componente più \textit{importante} da analizzare in quanto descrive la compressione/estensione della trave) può essere espressa come
		\begin{equation} \label{eq:sv:wnormale}
			 w(z) = \frac N{EA}z + c \qquad c \in \mathds R \textrm{ costante}
		\end{equation}
	\end{osservazione}
	\begin{osservazione}
		Dall'equazione \ref{eq:sv:camponormale} è possibile affermare che per una materiale standard (rapporto di Poisson $\nu$ compreso tra $0$ ed $\frac 1 2$) sottoposto a trazione ($N> 0$) la sua sezione diminuisce ($\exx, \eyy < 0$), mentre al contrario se è sottoposto a compressione ($N<0$) la sua sezione si espande ($\exx,\eyy > 0$). 
	\end{osservazione}
	\begin{concetto}
		E' possibile osservare che la \textbf{trasformazione} associata all'azione normale $N$ è \textbf{omotetica}, ossia mantiene invariato l'angolo relativo tra due segmenti qualsiasi prima e dopo trasformazione.		
	\end{concetto}
	\begin{dimostrazione}
		Noto che il tensore delle deformazioni $\E$ presenta solamente contributi diagonali $\exx,\eyy,\ezz$, scelta una generica retta radiale all'asse $z$ della trave di direzione $\vers r= \big(\cos\theta,\sin\theta,0\big)$, allora la sua deformazione lungo tale asse assume valore
		\[ \varepsilon_{rr} = \big(\E \vers r\big) \cdot \vers r = \exx \cos^2\theta + \eyy\sin^2\theta = - \nu \frac N{EA} \] 
		Questo dimostra dunque che la deformazione radiale $\varepsilon_{rr}$ è costante nella trave ed è indipendente dall'orientazione $\theta$ della retta $\vers r$ che si considera.
	\end{dimostrazione}

	E' possibile calcolare inoltre la variazione di volume considerando che l'elongazione volumetrica $\varepsilon_v$ è data dalla somma dei contributi delle deformazioni principali $\exx + \eyy + \ezz$ che porta al risultato
	\[ \varepsilon_v = \frac N{EA} ( 1 -2\nu) \qquad \Rightarrow \quad \Delta V = \int_V \varepsilon_v \, dV = \frac{NV}{EA} \big(1-2\nu\big) = \frac {NL} E \big(1-2\nu\big) \]
	dove il risultato finale si ottiene considerando la trave di volume $V$ come rettilinea di lunghezza assiale $L$ e sezione costante di area $A$.
	
	\begin{osservazione}
		Per \textbf{materiali standard} (rapporto di Poisson compreso tra 0 e $\frac 1 2$) è possibile affermare che la variazione volumetrica $\Delta V$ è sempre positiva per campioni sottoposti a trazione, e dunque si parla di materiali \textbf{dilatanti}. Nel caso particolare in cui $\nu = \frac 1 2$ si osserva che $\Delta V = 0$: la \textbf{deformazione} in questo caso è dunque \textbf{plastica}.
	\end{osservazione}

	\subsection{Rigidezza elastica equivalente}
		Si consideri come esempio la barra rettilinea mostrata in figura \ref{travevincolata}.
		\figura{6}{1}{travevincolata}{trave di lunghezza $L$ vincolata ad un estremo e caricata con azione normale $N$ all'estremo opposto.}{travevincolata}
		
		In questa situazione \textit{standard} il la componente $w$ del campo di spostamento $\ov u$ della barra è coerentemente modellato dall'espressione $w(z) = \frac N{EA}z + c$. Dalla scelta del sistema di riferimento si osserva che all'incastro, coordinata $z=0$, si deve ipotizzare spostamento nullo ($w=0$): questo permette di determinare che la costante $c$ è nulla e il campo di spostamento vale
		\[ w(z) = \frac N{EA}z  \]
		La variazione di lunghezza $\Delta L$ della trave può essere calcolata semplicemente valutando il campo di spostamento valutato nell'estremo opposto della trave, e dunque
		\[ \Delta L = w(L) = \frac{N}{EA}L \]
		
		\begin{concetto}
			Si osserva che la \textbf{trave} è \textbf{modellata} come una \textbf{molla lineare}; considerando infatti che l'azione normale $N$ coincide con la forza $F$ applicata sulla molla, allora l'allungamento della molla $\Delta x$ (associato alla variazione di lunghezza $\Delta L$) dipende dalla forza secondo un coefficiente $k$ dalla relazione
			\begin{equation}
				F = k\, \Delta x \qquad \leftarrow \quad k = \frac {EA} L
			\end{equation}
			dove il coefficiente $k$ prende il nome di \textbf{rigidezza elastica equivalente} della trave.
		\end{concetto}
		Considerando la trave come una molla, allora essa può essere equiparata ad un'\textbf{accumulatore di energia potenziale}; infatti è possibile dimostrare che l'energia immagazzinata da una molla (e dunque dalla trave) sottoposta ad una forza $F$ e allungamento $\Delta x$ secondo la legge
		\begin{equation}
			U_e = \frac 1 2 F\, \Delta x = \frac 1 2 N \, \Delta L = \frac 1 2 N \frac N{EA}L  = \frac 1 2 \frac{N^2}{EA}L
		\end{equation}
		\begin{osservazione}
			Per il teorema di Clapeyron è possibile osservare la relazione tra il lavoro associato alle azioni esterne e il lavoro interno dovuto alla deformazione elastica.
		\end{osservazione}
	
	\subsection{Verifica e dimensionamento}
		Nel caso si prova di pura trazione/compressione monoassiale, allora la tensione equivalente (per tutti i criteri) coincide con la tensione $\szz$ del sistema, e in particolare
		\[ \sigma_{eq} = \szz = \frac NA \]
		Posta la tensione critica $\sigma_{cr}$ del materiale e il coefficiente di sicurezza $\phi$ per il problema in equazione, la \textbf{verifica} si effettua invertendo la relazione $\sigma_{eq} \leq \sigma_{amm} = \sigma_{cr}\phi$, ottenendo dunque il risultato
		\[ N \leq \frac{\sigma_{cr}}{\phi} A \]
		
		Al contrario il \textbf{dimensionamento} si basa sul determinare l'area $A$ minima a supportare la tensione ammissibile; per fare questo è sufficiente manipolare l'equazione precedente ottenendo il risultato
		\[ A \geq \frac{N\phi}{\sigma_{cr}	} \]
		
		Le operazioni di verifica e dimensionamento possono essere anche effettuate mediante rappresentazione dello stato di tensione monoassiale sul piano di Mohr, come si osserva in figura \ref{mohrnormcrit}.
		
		\figura{4}{2}{mohrnormcrit}{rappresentazione dello stato di tensione nel piano di Mohr; aumentando la componente di azione normale $N$ lo stato di tensione associato (semicerchio blu) si espande fino a sovrapporsi al cerchio rappresentativo dello stato di tensione ammissibile (in rosso). }{mohrnormcrit}
		
	\subsection{Estensioni di validità}
		\paragraph{Estensioni geometriche} A questo punto è possibile cercare di capire quando è possibile estendere la soluzione dell'azione normale alle travi che non rispettano le ipotesi iniziali di Saint Venant. In particolare analizzando le \textbf{variazioni di sezione è possibile affermare che}:
		\begin{itemize}
			\item se la sezione \textbf{varia con regolarità} blandamente lungo l'asse, allora non è necessario introdurre una distanza di estinzione; noto dunque il valore dell'area $A(z)$ funzione della coordinata della trave è possibile ipotizzare che anche lo stato tensionale $\szz$ dipenda da $z$ secondo l'espressione
			\[ \szz (z) = \frac N {A(z)} \]
			In questa situazione decadrebbe la condizione di mantello scarico (le tensioni $\szz$ sulle areole esterne della sezione di fatto esercitano una forza sul mantello), tuttavia l'ipotesi di variazione blanda della sezione rende trascurabile tale effetto;
			
			\item se la sezione \textbf{varia bruscamente} è possibile utilizzare il modello di Saint Venant introducendo delle nuove  zone di estinzione e utilizzando le equazioni negli spezzoni di trave al di fuori delle stesse, come si può osservare in figura \ref{bruscvar}. In generale i 3 spezzoni ricavati possono avere anche proprietà del materiale diverse (quale il modulo elastico $E$). In questo caso la variazione di lunghezza complessiva  della trave può essere calcolata sommando la variazione di lunghezza dei singoli spezzoni:
			\[ \Delta L_{tot} = \Delta L_1 + \Delta L_2 + \Delta L_3 = N \sum_{i=1}^{3} \frac{L_i}{E_iA_i} \]
			
		\end{itemize}
		
		\figura{6}{2}{bruscvar}{trave con brusche variazioni di sezione; il modello di Saint Venant può essere utilizzato per coordinate $z$ distanti dalle sezioni $A$ e $B$ dove si ha la variazione brusca.}{bruscvar}
		
		\begin{concetto} \label{conc:sv:fattoreforma}
			Per valutare lo stato di tensione locale di una sezione che varia bruscamente si ricorre, in generale, ad un \textbf{fattore di forma} $K_t$ che permette di calcolare lo stato di tensione massimo $\sigma_{max}$ nella sezione che varia rispetto alla tensione nominale $\sigma_{zz,i}$ di uno dei due spezzoni rispetto ai quali si ha la variazione di sezione:
			\begin{equation}
				K_t = \frac{\sigma_{max}}{\sigma_{zz,2}} = f\left(\frac D d, \frac r d\right)
			\end{equation}
			In generale il fattore di forma, per una variazione di tubazione cilindrica, dipende dal raggio di raccordo $r$ delle sezioni e dai diamtri $D$ e $d$ delle due tubazioni cilindriche, come si osserva in figura \ref{fforma}.
		\end{concetto}
	
		\figura{8}{0.7}{fforma}{stati tensionali in una trave che varia sezione bruscamente.}{fforma}
		
		Lontano dalle zone di estinzione è dunque possibile pensare di utilizzare il modello di Saint Venant per determinare lo stato di tensione della trave (come, rispetto alla figura \ref{fforma}, nelle sezioni $a'$ e $b'$). In corrispondenza della variazione di sezione si ha il massimo picco di valore dello stato di tensione $\szz$ ed è per questo che si introduce il fattore di forma per modellare questa variazione rispetto al valore nominale durante la verifica/dimensionamento.
		
		In generale il fattore di forma è ottenuto sperimentalmente e i risultati vengono posti in opportuni diagrammi, come quello in figura \ref{fforma-b}.
		
		\figura{5}{1}{fforma-b}{fattore di forma $K_t$ in funzione dei rapporti $D/d$ e $r/d$.}{fforma-b}
		
		
		\paragraph{Estensioni al carico} Come è stato possibile effettuare delle estensioni sulla geometria della trave rispetto alle quali risulta possibile considerare valida la teoria di Saint Venant, analoghe estensioni possono essere fatte per quanto riguarda i \textbf{carichi}:
		\begin{itemize}
			\item nel caso di \textbf{carichi concentrati}, analogamente alle sezioni che variano bruscamente, si ricorre all'aggiunta di un'ulteriore zona di estinzione in prossimità del nuovo carico applicato; tramite il diagramma di azione interna è dunque poi possibile stabilire lo stato tensionale $\szz$ nei vari spezzone di trave, come nelle immagini che seguono.
			\begin{center}
				\includegraphics[width = 0.48 \linewidth]{fconc-a}
				\includegraphics[width = 0.35 \linewidth]{fconc-b}
			\end{center}
			
			\item la teoria di Saint Venant può essere considerata valida anche per \textbf{carichi distribuiti} lungo l'asse della trave (carichi che possono modellare, per esempio, le azioni di volume come la forza peso).
		\end{itemize}
		
		\begin{esempio}{: trave soggetta a carichi distribuiti}
			Si consideri la trave in figura come segue incastrata sulla superficie superiore sul soffitto.
			\begin{center}
				\includegraphics[width=4cm]{az-vol}
			\end{center}
			Della trave è possibile modellare l'azione di volume dovuta al peso come un'azione interna $n(z)$ distribuita di valore $\rho g A$. Per integrazione lungo l'asse è possibile calcolare sia la reazione $R$ in corrispondenza dell'incastro (che rende il sistema statico), sia l'azione interna $N$ in funzione della coordinata $z$ della trave:
			\[ R = \int_0^L n(z)\, dz = \rho g AL \qquad \Rightarrow \quad N(z) = R- \int_0^z n(z')\,dz' = \rho g A\big(L-z\big) = R \left(1 - \frac z L\right) \]
			
			Lo \textbf{stato di tensione} (e conseguentemente di \textbf{deformazione}) si può dunque determinare effettuando il rapporto dell'azione interna $N(z)$ in funzione della posizione nella trave e l'area $A$ della sezione della trave:
			\[ \szz(z) = \frac{N(z)}{A} = \frac RA\left(1- \frac z L\right)  \qquad \Rightarrow \qquad \ezz(z) = \frac{\szz(z)}{E} = \frac R{EA}\left(1- \frac z L\right) \]
			
			La \textbf{variazione di lunghezza} $\Delta L$ può essere determinata sempre tramite integrazione dell'elongazione $\ezz$, con il risultato che
			\[ \Delta L = \int_0^L \ezz(z)  \, dz = \int_0^L \frac{\rho g \big(L-z\big)}{E} \, dz = \frac{\rho gL}{2E} = \frac{RL}{2EA} \]
			L'energia potenziale elastica accumulata dalla trave può essere determinata per integrazione dei potenziali elastici del materiale lineare isotropo lineare:
			\[ U_e = \int_V \psi \, dV \quad \xrightarrow{\psi = \frac{\szz^2}{2E}}\quad \int_0^L \int_A \frac 1 {2E} \frac{R^2}{E^2} \left(1 - \frac z L\right)^2\, dZ\, dz = \frac{R^2L}{6EA} \]
		\end{esempio}

\section{Soluzione di problemi iperstatici}
	\begin{concetto}
		Dato un \textbf{problema} di \textbf{travi vincolate iperstaticamente}, è possibile risolvere lo stesso applicando in maniera sistematica il principio di sovrapposizione degli effetti.
	\end{concetto}
	Per spiegare come risolvere problemi iperstatici verranno mostrati 4 metodi esemplificati nel caso di presenza di sola azione normale $N$, ma i metodi possono essere applicati anche per tutte le altre analisi che verranno mostrate.
	
	In particolare i metodi mostrati si basano tutti sulla risoluzione del problema iperstatico mostrato in figura \ref{fig:sv:traveiperstatica-a}.
	
	\begin{figure}[bht]
		\centering
		\begin{subfigure}{0.48\linewidth}
			\centering
			\includegraphics[width=0.9\linewidth]{iperstat-a} \caption{}
		\end{subfigure}
		\begin{subfigure}{0.48\linewidth}
			\centering
			\includegraphics[width=0.9\linewidth]{iperstat-b} \caption{}
		\end{subfigure}
		\caption{trave vincolata iperstaticamente (a) e relativo schema di corpo libero (b).} \label{fig:sv:traveiperstatica-a}
	\end{figure}
	
	\subsection{Metodo delle forze}
		\begin{concetto}
			Il \textbf{metodo delle forze} si basa sulla sostituzione di un vincolo iperstatico con una cosiddetta \textbf{\textit{incognita iperstatica}} $X$; il metodo termina con la risoluzione nelle incognite iperstatiche per rendere congruenti i sistemi ricavati.
		\end{concetto}
		
		Facendo riferimento allo schema in figura \ref{fig:sv:traveiperstatica-a}, è possibile scegliere come incognita iperstatica $X$ la reazione $R_{Bz}$ assiale della trave nella cerniera $B$ e sostituendo la stessa con un pattino (che è libero di traslare).
		
		\figura{7}{0.7}{iperstat-c}{trave dell'esempio in cui viene sostituita la reazione $R_{Bz}$ con l'incognita iperstatica $X$.}{iperstat-c}
		
		A questo punto è necessario analizzare due sistemi distinti che dovranno poi essere posti congruenti per determinare l'incognita iperstatica:
		\begin{itemize}
			\item il primo sistema, pedice 0, considera che l'unica forza applicata è l'azione esterna $F$ applicata nel punto $C$; in questo caso il diagramma di azione interna è mostrato in questa figura:
			\begin{center}
				\includegraphics[width=6.5cm]{iperstat-c1}			
			\end{center}
			Osservando che lo stato di tensione $\szz$ è presente, uniformemente, solamente nel tratto $AC$, allora lo spostamento $w_{B0}$ del punto $B$ (dove è applicata l'incognita iperstatica) dipende solamente dalla deformazione del tratto $AC$; considerando che in tale tratto il campo di deformazione vale $\ezz = \frac F{EA}$ allora si determina
			\[ w_{B0} = \frac F{EA} a \] 
			
			\item considerando invece un secondo sistema (pedice 1) dove l'unica azione presente è associata all'incognita iperstatica $X$, facendo riferimento al diagramma di azione interna che segue si osserva che l'elongazione costante in tutto il tratto $AB$ della trave vale $\ezz = - \frac{X}{EA}$ (il segno negativo è dovuto al verso di azione di $X$) che, integrato su tutta la lunghezza dell'asta, determina un campo di spostamento del punto $B$ pari a 
			\[ w_{B1} = - \frac X {EA} \big(a + b\big) \]
			\begin{center}
				\includegraphics[width=6.5cm]{iperstat-c2}			
			\end{center}
			
		\end{itemize}
		Per via del principio di sovrapposizione degli effetti lo spostamento $w_B$ del punto $B$ è dato dalla somma dei contributi di spostamento del primo e del secondo sistema: $w_B = w_{B0} + w_{B1}$. Considerando che il vincolo iperstatico impone che lo spostamento del punto $B$ sia nullo  ($w_B = 0$), allora è possibile risolvere il sistema per determinare l'incognita iperstatica $X = R_{Bz}$ e conseguentemente la reazione nel punto $A$:
		\[ \underbrace{\frac F{EA} a - \frac X{EA} (a+b)}_{w_{B0}+w_{B1}} = 0 \qquad \Rightarrow \quad X = R_{Bz}= F \frac a {a+b} \quad,\quad R_{Az} = F - X = F \frac b{a+b}   \] 
		
		\begin{osservazione}
			Tra tutti i sistemi questo è il più \textit{intuitivo} in quanto si cerca tra tutti gli infiniti valori della reazione incognita l'unico che produce una soluzione congruente.
		\end{osservazione}
	
	\subsection{Metodo degli spostamenti}
		\begin{concetto}
			Il \textbf{metodo degli spostamenti} si basa sul cercare le azioni che rendono plausibile lo spostamento di un punto preciso della trave.
		\end{concetto}
		Facendo riferimento all'esempio in figura \ref{fig:sv:traveiperstatica-a}, l'incognita iperstatica può essere associata allo spostamento $w_C$ del punto $C$ riferito sia ai tratti $AC$ che $CB$; è inoltre necessario imporre che, all'equilibrio, $F= R_{Az} + R_{Bz}$.
		
		Facendo riferimento allo schema di azione interna della trave (fig. \ref{iperstat-d1}) è possibile determinare lo stato di tensione nei due spezzoni che compongono la trave ipotizzando la deformazione $\ezz$ dovuta allo spostamento $w_C$ del punto $C$. In particolare considerando il primo tratto $AC$ si può osservare che
		\[ \varepsilon_{zz,AC} = \frac {w_C} a \quad \rightarrow \quad \sigma_{zz,AC} = E \frac {w_C}a \quad \rightarrow \quad N_{AC} = AE \sigma_{zz,AC} =AE\frac{w_C}a = R_{Az}   \]  
		Le stesse considerazioni possono anche essere estese al secondo tratto di trave $CB$, determinando il nuovo stato di tensione al suo interno che risulta valere
		\[ \varepsilon_{zz,CB} = -\frac {w_C} b \quad \rightarrow \quad \sigma_{zz,CB} = -E \frac {w_C}b \quad \rightarrow \quad N_{CB} = -AE \sigma_{zz,CB} =AE\frac{w_C}a = -R_{Bz}   \]  
		\figura{6}{1}{iperstat-d1}{schema di corpo libero e di azione interna per il metodo degli spostamenti.}{iperstat-d1}
		
		In condizioni statiche stazionarie, come affermato in precedenza, è necessario imporre l'equilibrio delle forze: questo permette dunque di esplicitare lo spostamento $w_C$ del punto $C$ in funzione dei parametri del problema:
		\[ F = R_{Az} + R_{Bz} = AE \frac{w_C}{A} + AE \frac{w_C}{b} \qquad \Rightarrow \quad w_C = \frac F{AE} \frac{ab}{a+b} \]
		Sostituendo il risultato ottenuto per $w_C$ nelle equazioni determinate in precedenza, è possibile esplicitare il valore delle reazioni vincolari nei punti $A$ e $B$ che risultano valere
		\[ R_{Az} = AE\frac{w_C}{a} = F \frac{b}{a+b} \qquad R_{Vz} = AE\frac{w_C}{b} = F \frac{a}{a+b} \]
		
		\begin{osservazione}
			Questo metodo è meno intuitivo in quanto cerca tra gli infiniti valori dello spostamento $w_C$ quello che produce l'unica soluzione equilibrata; questo metodo è tuttavia comodo in quanto più facilmente implementabile numericamente.
		\end{osservazione}

	\subsection{Metodo della deformata elastica}		
		\begin{concetto}
			Il \textbf{metodo della deformata elastica} si basa sulla sostituzione dei vincoli iperstatici con delle rispettive incognite iperstatiche $X$ in modo da rendere il sistema isostatico. Per risolvere il sistema si procede dunque a bilanciare la deformata elastica in alcuni punti particolari.
		\end{concetto}
		Considerando lo schema di esempio in figura \ref{fig:sv:traveiperstatica-a}, sostituendo alla reazione $R_{Bz}$ sovrabbondante l'incognita iperstatica $X$ si ottiene uno schema di corpo libero e di azione interna come in questa figura:
		\begin{center}
			\includegraphics[width=6cm]{iperstat-e}
		\end{center}
		Considerando che nel primo tratto di trave $AC$ l'azione interna $N_1$ vale $F-X$, per sostituzione è possibile ricavare il campo di spostamento $w_1$ nel primo tratto della trave (eq. \ref{eq:sv:wnormale}, pag. \pageref{eq:sv:wnormale}) che risulta valere
		\[ w_1(z) = \frac {N_1}{EA}z + c_1 = \frac{F-X}{EA}z  + c_1 \]
		Per determinare il coefficiente di integrazione $c_1$ è sufficiente considerare che il valore del campo di spostamento nel punto $A$ ($z=0$) debba essere nullo ($w_1(0)=0$): questo permette di affermare che la costante $c_1$ deve essere nulla. Analogamente per il secondo tratto di trave $CB$ il campo di spostamento è determinato come
		\[ w_2(z) = \frac{N_2}{EA}z +c_2  = \frac{-X}{EA}z +c_2  \]
		Dovendo essere il punto $B$ ($z=a+b$) fisso per via dell'iperstaticità del problema, è necessario che anche il relativo campo di spostamento sia nullo: questo permette di determinare la costante di integrazione $c_2$ che risulta valere $\frac{X}{EA}(a+b)$. A questo punto per risolvere il problema è sufficiente imporre che gli spostamenti $w_1,w_2$ nel punto $C$ ($z=a$) abbiano lo stesso valore e dunque:
		\[ w_1(a) = w_2(a)  \qquad \rightarrow \qquad \frac{F-X}{EA} z = \frac{X}{EA} \big(a+b-z\big) \]
		\[ \Rightarrow \qquad X = R_{Bz} = F \frac a {a+b} \qquad R_{Az} = F-X = F \frac b {a+b} \]
		
	\subsection{Metodo di Castsigliano}
		\begin{concetto}
			Il \textbf{metodo di Castigliano} si basa sull'omonimo teorema energetico e anche in questo caso (come per il metodo delle forze/deformata elastica) si ricorre alla sostituzione delle reazioni sovrabbondanti con incognite iperstatiche $X$.
		\end{concetto}
		Noto che per il teorema di Castigliano lo spostamento di direzione $\ov \eta_i$ associato ad una forza generalizzata $\ov Q_i$ coincide con la derivata parziale dell'energia potenziale elastica rispetto alla forza stessa ($\partial U_e / \partial \ov Q_i = \ov \eta_i$), allora è possibile sfruttare tale principio imponendo che la derivata parziale rispetto all'incognita iperstatica $X$ dell'energia potenziale sia nulla in quanto non si deve avere spostamento del punto $B$ che è vincolato iperstaticamente.
		
		In primo luogo è dunque necessario calcolare l'energia potenziale elastica $U_e$ della trave considerando che essa è composta da due spezzoni $AC$ e $CB$ con stati tensionali distinti e dunque:
		\[  U_e = \frac 1 2 \frac{N_1^2}{EA}a + \frac 1 2 \frac{N_2^2}{EA}b = \frac 1 2 \frac{\big(F-X\big)^2 a + X^2 b}{EA}  \]
		Come appena affermato, per risolvere il problema è sufficiente porre nulla la derivata dell'energia potenziale $U_e$ rispetto a $X$, risolvendo per l'incognita iperstatica:
		\[ \pd {U_e}X = \frac{\big(F-X) a (-1) + Xb\big)}{EA} = 0 \qquad \Rightarrow \qquad X = R_{Bz} = F\frac a {a+b} \quad R_{Az} = F-X = F \frac b {a+b} \]
	
	\subsection*{Considerazioni finali}
		Appurato che le soluzioni del problema sono composte dalle reazioni $R_{Az} = F \frac b {a+b}$ e $R_{Bz} = F \frac a {a+b}$ è possibile calcolare lo stato di tensione interno $\szz$ (dal quale consegue lo stato di deformazione $\ezz$) e conseguentemente il campo di spostamento nei due spezzoni di trave:
		\begin{align*}
			&& \textrm{tratto } AC \qquad \sigma_{zz,1} & = \frac F A \frac b{a+b} & w_1(z) &= \frac F{EA} \frac b {a+b} z \\
			&& \textrm{tratto } CB \qquad \sigma_{zz,2} & = \frac F A \frac a{a+b} & w_2(z) &= \frac F{EA} \frac a {a+b} \big(a+b-z\big) 
		\end{align*}
		E' comunque possibile osservare come entrambi i campi di spostamento $w_1,w_2$ abbiano valore uguale nel punto di \textit{contatto} $C$ (coordinata $z=a$) pari a valore $\frac F{EA} \frac{ab}{a+b}$; il campo di spostamento in funzione della coordinata assiale è mostrato in figura \ref{iperstat-f}.
		\figura{5}{1.2}{iperstat-f}{campo di spostamento $w(z)$ in funzione della coordinata $z$ dell'asse.}{iperstat-f}
		
		\textbf{Considerazioni energetiche} possono essere effettuate utilizzando il teorema di Clapeyron per il quale la metà del lavoro delle forze esterne coincide con l'energia elastica immagazzinata dal corpo; considerando infatti che la sola posizione finale del corpo, l'unico lavoro è dovuto alla forza $F$ applicata nel punto $C$ per la quale
		\[ L^\textrm{ext} = F\, w(a) = \frac{F^2}{EA} \frac{ab}{a+b} \]
		Potendo calcolare per integrazione l'energia potenziale elastica della trave, si verifica l'uguaglianza formulata da Clapeyron:
		\[ U_e = \int_V \psi\, dV = \frac 1 2 \frac{N_1^2}{EA}a + \frac 1 2 \frac{N_2^2}{EA}b = \frac{F^2}{2EA} \left(\frac{b^2a}{(a+b)^2} + \frac{a^2b}{(a+b)^2} \right) = \frac 1 2\frac{F^2}{EA} \frac{ab}{a+b} = \frac 1 2 L^\textrm{ext} \]
	
	\subsection{Effetti della temperatura}
		Un aumento di temperatura $\Delta T$ rispetto ad una situazione iniziale provoca una dilatazione che, tramite il \textbf{coefficiente di dilatazione termica lineare} $\alpha$, determina uno stato di deformazione $\exx = \eyy = \ezz = \alpha\, \Delta T$. Questo significa che se la differenza di temperatura $\Delta T$ è uniforme su ogni sezione, allora le stesse si mantengono piane e lungo la direzione assiale si ha un'\textbf{espansione/contrazione omotetica} (senza variazione di angoli) che può essere misurata come
		\[ \Delta L = \int_0 ^L \ezz \, dz = \int_0^L \alpha(z) \, \Delta T(z) \, dz \]
		Nel caso particolare in cui il coefficiente $\alpha$ sia indipendente dalla posizione e che la differenza di temperatura $\Delta T$ sia uguale per ogni sezione, l'espressione precedente si riduce all'espressione $\Delta L = \alpha\, \Delta T\, L$.
		
		\figura{6}{1}{iperstat-temp}{schema, diagramma di corpo libero e di azione interna di una trave vincolata iperstaticamente soggetta ad una variazione di temperatura.}{iperstat-temp}
		
		Considerando ora l'esempio di una trave vincolata iperstaticamente soggetta ad una variazione di temperatura, come in figura \ref{iperstat-temp}, è possibile pensare di determinare le reazioni vincolari utilizzando il metodo della deformata elastica. Assegnando alla reazione $R_{Bz}$ l'incognita iperstatica $X$, lo spostamento $w_{B,X}$ del punto $B$ dovuto a tale azione (ricavato dall'equazione \ref{eq:sv:wnormale}, pag. \pageref{eq:sv:wnormale}) è determinato dall'equazione
		\[ w_{B,X} = - \frac X{EA}L \]
		Lo spostamento dovuto all'aumento di temperatura, come osservato in precedenza, risulta valere $w_{B,\Delta T} = \alpha\, \Delta T \, L$; applicando il principio di sovrapposizione degli effetti il campo di spostamento totale è dato dalla somma dei contributi dovuti all'azione $X$ e alla temperatura $\Delta T$; dovendo, per via del vincolo nel punto $B$, essere nullo il campo di spostamento in tale posizione è possibile determinare l'incognita iperstatica $X$ e il conseguente stato di tensione:
		\[  w_{B,tot} =  w_{B,X} + w_{B, \Delta T} = - \frac X{EA}L + \alpha\, \Delta T \, L \qquad \Rightarrow X = EA \alpha\, \Delta T \qquad \Rightarrow  \szz = -E\alpha\,\Delta T  \]
		Questo stato di tensione $\szz$ è dunque dato solamente dall'aumento di temperatura $\Delta T$ del corpo (in quanto si ipotizza che per $\Delta T=0$ lo stato di tensione sia nullo).
		
		\vspace{3mm}
		
		Lo stesso problema si poteva risolvere utilizzando il metodo di Castigliano, considerando che il potenziale $\psi$ può essere espresso come $\frac 1 2 E \ezz$. In questo caso lo stato di deformazione $\ezz$ è dovuto a due contribuiti: quello dell'incognita iperstatica $X$ e quello dovuto all'aumento di temperatura
		\[ \ezz = - \frac X{EA} + \alpha\, \Delta T \]
		
		Potendo calcolare l'energia potenziale elastica $U_e= \int_v\psi\, dV$ tramite integrazione, applicando il metodo di castigliano  derivando tale energia rispetto all'incognita iperstatica $X$ (e ponendo il risultato pari a zero in quanto nel punto $B$ non si deve avere spostamento), è possibile ottenere lo stesso risultato ottenuto con il metodo degli spostamenti:
		\[ \pd{U_e}X = EA \int_L \left(- \frac X {EA} + \alpha\,\Delta T\right)\left(-\frac 1 {EA}\right)\, dz = 0 \qquad \Rightarrow \quad X = EA\alpha\, \Delta T \]
	
	\subsection{Equazione di Poisson}
		Una trave sottoposta a carico assiale distribuito $n(z)$ non presenta più un'azione normale $N(z)$ costante lungo la trave; analizzando infatti un concio elementare (rappresentato in figura \ref{sv-concio-poisson}) è possibile bilanciare le forze agenti sullo stesso arrivando alla conclusione che
		\begin{equation} \label{eq:sv:temp1}
			N + \frac{dN}{dz}dz - N + n(z) = 0 \qquad \Rightarrow \quad \frac {dN} {dz} = -n(z)
		\end{equation}
		\figura{4}{1.5}{sv-concio-poisson}{azioni sul concio elementare di una trave soggetta a carico distribuito.}{sv-concio-poisson}
		
		In generale l'azione normale $N(z)$, riarraniagiando l'equazione \ref{eq:sv:wnormale} (pag. \pageref{eq:sv:wnormale}) dipende sia dal modulo di Young $E$ che dall'area $A(z)$ della sezione (parametri che lungo la trave possono variare in generale), e dunque
		\[ dN = E(z)A(z) \frac {dw}{dz}\]
		\begin{concetto}
			Considerando un caso particolare in cui sia il modulo $E$ che l'area $A$ sono costanti lungo la trave, invertendo opportunamente l'espressione appena ricavato è possibile riscrivere l'equazione \ref{eq:sv:temp1} per ottenere la cosiddetta \textbf{equazione di Poisson}:
			\begin{equation}
				\frac{dN}{dz} \frac d {dz} \left(EA \frac{dw}{dz}\right) = - n(z) \qquad \Rightarrow \quad \frac{d^2w}{dz^2} = - \frac{n(z)}{EA}
			\end{equation}
		\end{concetto}
		Noto dunque il profilo della forza distribuita $n(z)$ lungo la trave tramite doppia integrazione è possibile determinare direttamente la componente $w$ del campo di spostamento della trave; nel caso particolare di carico distribuito costante $n(z) = n_0$, per integrazione è possibile dimostrare che
		\[ EA\, w = -n_0 \frac {z^2}2 + c_1 z + c_2\]
		dove le costanti $c_1,c_2$ dipendono dai vincoli sulla trave (per esempio se essa è vincolata, lo spostamento deve essere nullo, se essa è libera la sua derivata deve essere nulla).
	
\section{Analisi della flessione}
	Osservando l'equazione \ref{eq:sv:szznormale} (pag. \pageref{eq:sv:szznormale}) è possibile osservare che la tensione scalare $\szz$ (e conseguentemente lo stato di deformazione) dipende da contribuiti di azione normale $N$ (descritta a partire da pag. \pageref{sec:sv:normale}) e da dei contributi dipendenti dai \textbf{momenti flettenti} $M_x$ ed $M_y$ agenti sulla trave; essi risultano essere dipendenti dalla posizione sulla trave in quanto l'equazione originaria prevede una formulazione
	\[\szz = \frac NA + \frac \Mx \Ixx y - \frac \My \Iyy x \]
	\begin{concetto}
		Quando si ha l'applicazione di un solo momento flettente ($M_x = 0$ con $M_y\neq 0$ oppure $M_x\neq 0$ con $M_y=0$) si parla di \textbf{flessione retta}, mentre se entrambe le componenti di momento sono applicate si parla di \textbf{flessione deviata}.
	\end{concetto}
	\begin{osservazione}
		Per via della simmetricità delle formulazioni per la flessione retta dovuta ad $M_x$ e ad $M_y$, è sufficiente studiare uno solo dei due casi e, per dualità, determinare i risultati associati all'altro caso.
	\end{osservazione}
	
	\subsection{Analisi della flessione retta}
		\begin{concetto}
			Considerando la sola flessione retta determinata dal momento $M_x$ ($M_y$), l'\textbf{equazione di Navier} (ripresa dall'equazione \ref{eq:sv:szznormale}, pag. \pageref{eq:sv:szznormale}) permette di determinare lo stato di tensione di sola componente $\szz$ (e relativo stato di tensione) come
			\begin{equation} \label{eq:sv:statoflettente}
				\szz = \frac \Mx \Ixx y \qquad \ezz = \frac \Mx {E\Ixx} y \qquad \exx = \eyy = -\nu \frac \Mx{E\Ixx} y
			\end{equation} 
		\end{concetto}
		\begin{osservazione}
			Nel caso di presenza di solo stato momento flettente $M_y$ ($M_x=0$) gli stati di tensione/deformazioni risultanti sarebbero valsi
			\[ \szz = -\frac \My \Iyy x \qquad \ezz =-\frac \My {E\Iyy} x \qquad \exx = \eyy = \nu \frac \My{E\Iyy} x \]
		\end{osservazione}
		\begin{concetto} \label{conc:sv:asseneutro}
			Il campo scalare $\szz$ è lineare lungo la direzione $y$:; in particolare pungo la retta orizzontale $x$ (di equazione $y=0$) lo stato di tensione è nullo: tale direzione prende il nome di \textbf{asse neutro}.
		\end{concetto}
	
		\paragraph{Prove sperimentali} Un modo per testare sperimentalmente la risposta a pura flessione retta è quella di effettuare una \textbf{prova di flessione a 4 punti}, mostrata in figura \ref{fless4punti}. Un altra prova che si può utilizzare per analizzare la flessione è quella di incernierare un provino ad un estremo e applicare una coppia di forze sull'altra estremità, come in figura \ref{provaflessione}.
		
		\figura{6}{1}{fless4punti}{prova di flessione a 4 punti.}{fless4punti}
		\figura{6}{1}{provaflessione}{prova di flessione ottenuta tramite applicazione di una coppia di forze.}{provaflessione}
		
	\subsubsection{Campo di spostamento}
		Lo stato di deformazione di una trave sottoposta a momento flettente è già stato analizzato in precedenza in equazione \ref{eq:sv:statoflettente}; inoltre per le ipotesi iniziali poste da Saint Venant è altresì possibile affermare che le componenti di deformazione angolare $\gamma_{xy}, \gamma_{xz},\gamma_{yz}$ sono nulle.
		
		Considerando il \textbf{concio elementare}, essendo la tensione $\szz$ distribuita in modo triangolare rispetto all'asse $y$ l'elemento infinitesimo tende a contrarsi nella zona in cui la tensione è negativa, mentre tende ad estendersi dove la tensione è positiva; lungo l'asse orizzontale $x$, essendo $\szz = 0$, la deformazione complessiva deve essere nulla.
		
		\figura{4}{1}{flettconcioelem}{concio elementare deformato da un momento flettente $M_x$.}{flettconcioelem}
		
		\begin{concetto}
			Si ipotizza dunque che un concio elementare sottoposto a solo momento flettente $M_x$ tende a \textbf{deformarsi} in modo da diventare un \textbf{arco di circonferenza} di \textbf{curvatura} $k_x = 1/R_x$ (dove $R_x$ è il \textbf{raggio di curvatura}). L'angolo al centro della circonferenza associato al concio deformato è generalmente indicato $d\theta_x$.
		\end{concetto}
		
		Facendo riferimento al segmento orizzontale $AB = dz = ds =R_x\, d\theta_x$ posto ad una distanza $y$ dall'asse $x$, come in figura \ref{flettconcioelem}, è possibile determinare la lunghezza che avrà tale segmento dopo deformazione e dunque l'\textbf{elongazione specifica} $\ezz$ rispetto all'asse $z$:
		\begin{align*}
			A'B' & = \big(R_x + y\big) d\theta_x \\
			\Rightarrow \quad \ezz  & = \frac{A'B'-AB}{AB} = \frac{\big(R_x+y\big) d\theta_x - dz }{dz} = \frac{\big(R_x + y\big)d\theta_x - R_x \, d\theta_x}{R_x \, d\theta_x} \\
			& = \frac y {R_x}
		\end{align*}
		Unendo questo risultato con l'espressione dello stato di deformazione $\ezz$ mostrato in equazione \ref{eq:sv:statoflettente} (pag. \pageref{eq:sv:statoflettente}) è possibile determinare in modo esplicito la curvatura (e il rispettivo raggio) del concio elementare in funzione dei parametri del materiale e del carico applicato:
		\begin{equation}
			\ezz = \frac y {R_x} =\frac \Mx{E\Ixx} y \qquad \Rightarrow \quad k_x = \frac \Mx{E\Ixx} \qquad R_x = \frac {E\Ixx} \Mx
		\end{equation}
		\begin{concetto}
			L'\textbf{ipotesi deformativa di Eulero-Bernoulli} ci permette di affermare che se il momento flettente applicato $M_x$ è costante lungo tutta la trave, allora lo stato di deformazione $\ezz$ è uguale in ogni tratto elementare di trave e dunque la trave deformata presenta una curvatura costante.
		\end{concetto}
		\begin{concetto}
			Integrando le equazioni di congruenza (\textbf{RIFERIMENTO}) è possibile determinare il \textbf{campo di spostamento} $\ov u(u,v,w)$ che, a meno di uno spostamento rigido arbitrario, risulta valere
			\begin{equation} \label{eq:sv:campoflessione}
			\ov u = \left\{ 
			\begin{split}
				u &= -\nu \, U(x,y) \\
				v &= -\frac \Mx {2E\Ixx} z^2 - \nu \, V(x,y) \\
				w &= - \frac{\Mx}{2E\Ixx} zy
			\end{split}\right.
			\qquad \leftarrow \left\{ 
			\begin{split}
				U(x,y) & = \frac{\Mx}{E\Ixx} xy \\
				V(x,y) & = \frac{\Mx}{E\Ixx} \big(y^2 - x^2\big) \\
			\end{split}\right.
			\end{equation}
			I termini indipendenti dal rapporto di Poisson $\nu$ sono nominati \textbf{campo primario}, mentre gli altri termini dipendenti da $\nu$ ($U,V$) rappresentano il \textbf{campo secondario}.
		\end{concetto}
		
		\paragraph{Analisi del campo di spostamento primario} Essendo il corpo in analisi una trave, allora è possibile affermare che in generale la coordinata assiale $z$ sia molto più grande delle coordinate delle sezioni ($z\gg x,y$): questo permette di affermare che nel campo di spostamento molto rilevante è il campo primario (che dipende da $z$ direttamente), mentre il campo di spostamento secondario è meno impattante e di fatto permette di definire l'\textit{inclinazione} della sezione deformata rispetto a quella originaria.
		
		A tal proposito è dunque possibile definire il \textbf{piano di flessione} appartenente al piano $y,z$ (dovuto alle componenti $v,w$ del campo di spostamento primario) rispetto al quale si ha la deformazione dell'asse della trave. In particolare $v$ rappresenta la traslazione dell'asse mentre $w$ modella l'effetto della rotazione della sezione che, per le ipotesi di Eulero-Bernoulli, deve essere sempre perpendicolare all'asse.
		
		\vspace{3mm}
		A questo punto è anche possibile calcolare l'angolo $\theta_x$ che la sezione della trave forma con l'asse verticale in funzione della coordinata $z$ tramite le equazioni di congruenza applicate al campo scalare definito in equazione \ref{eq:sv:campoflessione}:
		\[ \theta_x = \left( \pd w y - \pd v z \right) = \frac \Mx{E\Ixx} z \]
		\begin{concetto}
		    Essendo  la pendenza dell'asse della trave misurata dalla relazione $\partial v / \partial z = -\theta_x = - \frac{\Mx}{E\Ixx} z$, tramite integrazione è possibile osservare che l'\textbf{asse deformato} assume una \textbf{forma parabolica}:
    		\[ v = - \frac{\Mx}{2E\Ixx} z^2 + c \qquad c \in \mathds R \]
		\end{concetto}
		\begin{osservazione}
		    Rispetto all'ipotesi iniziale di deformata assiale circolare, questo risultato potrebbe sembrare incoerente, tuttavia per piccoli gradienti del campo di spostamento è possibile approssimare le due soluzioni come equivalenti.
		\end{osservazione}
		\begin{dimostrazione}
		    L'osservazione appena riportata può essere dimostrata considerando l'angolo $\theta_x$ sarebbe più correttamente modellato dall'espressione $ \arctan \big( - \partial v / \partial z\big)$. A questo punto è possibile calcolare la curvatura della trave come
		    \begin{align*}
		        k_x &=\frac{d\theta_x}{ds} = \frac{d\theta_x}{dz} \frac{dz}{ds} = \frac d {dz} \left[ \arctan \left( - \pd v z \right)\right] \frac{dz}{ds} \\
		        &= \frac{- \frac{d^2v}{dz^2}}{\Big[ 1 + 
		         \big(dv/dz\big)^2 \Big]^{3/2}}
		    \end{align*}
		    Nell'ipotesi di piccoli gradienti del campo di spostamento il termine $dv/dz \ll 1$ può essere trascurato: in questo caso il denominatore si riduce all'unità e la curvatura può essere approssimata dall'espressione 
		    \[ k_x \approx = - \pd{^2v}{z^2} = -\frac{\Mx}{E\Ixx} \]
		    ossia ad un arco di parabola, come si voleva dimostrare.
		\end{dimostrazione}
		
		\paragraph{Analisi del campo di spostamento secondario} Il campo di spostamento secondario modella la variazione della sezione della trave sottoposta a momento flettente ed è composto dalle componenti $u = - \nu \frac \Mx {E\Ixx}xy$ e $v = - \nu \frac{\Mx}{E\Ixx} (y^2-x^2)$. \\
		Considerando l'esempio di una trave rettangolare di dimensione $b\times h$, allora lungo l'asse verticale $x=0$ ($\forall y$) il campo $u$ è nullo, mentre agli estremi della trave (coordinate $x=\pm \frac b2$) il relativo campo di spostamento vale
		\[ u = \pm \nu \frac{\Mx}{E\Ixx} \frac b 2 y   \]
		
		Considerando invece l'asse orizzontale di equazione $y=0$, allora il campo di spostamento $v$ descrive un arco di parabola che, come nel caso precedente, può essere approssimato ad un arco di circonferenza di curvatura $k_y$ pari a:
		\begin{equation} \label{eq:sv:curvantielastica}
			v = -\nu \frac{\Mx}{E\Ixx} \big(-x^2\big) \qquad \Rightarrow \qquad k_y = \frac 1 {\rho_y} = - \nu \frac{\Mx}{E\Ixx}
		\end{equation}
		\begin{concetto}
			Il coefficiente $k_y$ in equazione \ref{eq:sv:curvantielastica} prende il nome di \textbf{curvatura antielastica} e il \textbf{raggio} $\rho_y$ rappresenta la coordinata sull'asse $y$ rispetto alla quale la sezione sembra diventare un arco di circonferenza.
		\end{concetto}
		\begin{nota}
			Essendo in generale $\nu \simeq 0.33$ per molti materiali, allora è possibile osservare che il raggio di curvatura $\rho_y$ è circa $\frac 1 3$ del raggio di curvatura $R_x$ dell'asse della trave, quindi l'effetto è molto meno evidente.
		\end{nota}
	
		\subsubsection{Flessione retta associata sull'asse y}
			Analizzando la flessione retta associata all'azione del solo momento $M_y$ ($M_x=0$), i cui stati di tensione/deformazione sono descritti dalle equazioni $\szz = - \frac \My \Iyy x$, $\ezz = \szz/E$, $\exx=\eyy = \nu \frac{\My}{E\Iyy}x$, allora l'asse neutro associato a tale flessione è quello verticale di equazione $x = 0$.
			
			Analogamente si determinano le curvature dell'asse e l'equazione della deformata elastica che misura lo spostamento nella componente $u$ come
			\[ k_y = \frac 1 {R_y}  = - \frac{\My}{E\Iyy} \qquad \frac{d^2u}{dz^2} = \frac{\My}{E\Iyy}  \]
			Il campo di spostamento $\ov u = \big(u,v,w\big)$ associato a tale flessione è espresso dalle componenti
			\[ u = \frac{\My}{2E\Iyy} z^2 + \nu \frac{\My}{2E\Iyy} \big(x^2-y^2\big) \qquad v = \nu \frac{\My}{E\Iyy}xy \qquad w = - \nu \frac \My{E\Iyy} xz \]
			
			
			
	\subsection{Verifica e dimensionamento, modulo di resistenza}
		Come affermato nel concetto \ref{conc:sv:asseneutro} (pag. \pageref{conc:sv:asseneutro}) l'asse neutro è quello verticale (di equazione $y=0$) ed è possibile osservare come esso sia baricentrico (per via della scelta iniziale del sistema di riferimento). Per effettuare dunque le operazioni di verifica e dimensionamento, note le tensioni $\sigma_{cr}^+,\sigma_{cr}^-$ del materiale, è sufficiente calcolare la distanza massima della sezione del corpo dall'asse neutro, in quanto essa rappresenta il punto di massima tensione $\szz$.
		
		Per effettuare le operazioni di verifica (e per analogia di dimensionamento) noto il fattore di sicurezza $\phi$ è sufficiente verificare le relazioni
		\begin{align*}
			\max \big\{ \szz \quad y \in \big[y_{min}, y_{max}\big]\big\} & \leq \frac{\sigma_{cr}^+}{\phi} \\
			\min \big\{ \szz \quad y \in \big[y_{min}, y_{max}\big]\big\} & \geq  \frac{\sigma_{cr}^-}{\phi}
		\end{align*}
		
		Note dunque le coordinate estremali $y_0, y_1$ delle sezioni è possibile pensare di calcolare gli stati di tensione massimi utilizzando tali valori:
		\[ \sigma_{zz,max} = \frac \Mx \Ixx y_1 \qquad \sigma_{zz,min} = \frac \Mx \Ixx y_0 \]
		L'idea sarebbe quella dunque di trovare un parametro che permette di \textit{riassumere} le informazioni geometriche sulla trave e permetta di determinare più agilmente lo stato di tensione massimo/minimo sulla sezione della trave.
		
		\begin{concetto}
			Si definisce il rapporto $W_x = \Ixx / y_{cr}$ tra momento di inerzia $\Ixx$ e la coordinata critica della sezione $y_{cr}$ il \textbf{modulo di resistenza}; potendo tabulare tale parametro per molte sezioni, le operazioni di verifica/dimensionamento vengono facilitate in quanto per calcolare la tensione massima critica sulla sezione è sufficiente effettuare l'operazione
			\begin{equation}
				\sigma_{zz,cr} = \frac{\Mx}{W_x}
			\end{equation}
		\end{concetto} 
		\begin{osservazione}
			Per materiali asimmetrici per i quali la tensione critica a compressione è diversa da quella a trazione, è necessario specificare due valori di moduli di resistenza diversi.
		\end{osservazione}
		\begin{esempio}{: modulo di resistenza di una sezione rettangolare} \label{es:sv:modresrettangolo}
			Una sezione rettangolare di dimensioni $b\times h$ ($b$ sull'asse orizzontale delle $x$) ha momento di inerzia $\Ixx$ noto pari a $\frac 1 {12} bh^3$. La coordinata $y_{cr}$  della sezione, per costruzione, è possibile osservare che vale $ y_{cr} = \pm \frac h 2$: questo permette dunque di stabilire il \textbf{modulo di resistenza} della sezione rettangolare come
			\[ W_x = \frac{\Ixx}{y_{cr}} = \frac 1 6 bh^2 =  \frac 1 6 A h  \]
		\end{esempio}
		\begin{esempio}{: modulo di resistenza di una sezione circolare}
			Analogamente al caso precedente, di una sezione circolare di diametro $d$ è noto sia l'inerzia $\Ixx =\frac \pi {64}d^4$, sia la distanza critica dall'asse neutro $y_{cr} = \frac d 2$ che permette di ricavare il modulo di resistenza della sezione circolare come
			\[ W_x = \frac{\Ixx}{y_{cr}} = \frac \pi {32} d^3  \]
	\end{esempio}
		
		Facendo riferimento all'esempio \ref{es:sv:modresrettangolo}, è possibile osservare che una scelta diversa della geometria della sezione può comportare un miglioramento della risposta della trave al momento flettente. In particolare, a parità di area $A$, aumentando la distanza del materiale dall'asse neutro (aumento di $h$) è possibile aumentare il modulo di resistenza $W_x$:
		\[ W_x \propto h  \]
		\begin{nota}
			Questo concetto può essere esteso solamente entro certi limiti, in quanto per esempio la base $b$ non può essere \textit{troppo piccola}; inoltre aumentando il modulo di resistenza $W_x$, contestualmente diminuisce il modulo di resistenza $W_y = \frac 1 6 A b$ (a parità di area $A$ di materiale usato per la sezione).
		\end{nota}
		Un modo per massimizzare il modulo di resistenza è quello di incrementare il momento di inerzia $\Ixx$ spostando l'area il più possibile dall'asse neutro $x$; la situazione limite di questo concetto è mostrata in figura \ref{sv-maxwx} dove l'area $A$ della sezione viene separata in due metà che possono essere poste ad una distanza $h$ dall'asse neutro arbitrariamente elevata.
		
		Tuttavia questa soluzione non è la migliore in quanto di fatto si ha la divisione della trave in due sottotravi, la cui faccia soggetta a compressione è particolarmente instabile (in quanto tenderebbe a deformarsi \textit{come un foglio di carta}).
		
		\figura{5}{1.5}{sv-maxwx}{sezione che, a parità di area $A$, permette di massimizzare $W_x$.}{sv-maxwx}
		
		Una soluzione tecnologica per ovviare a questo tipo di problema è dunque di congiungere queste due sezioni con un elemento verticale centrale, in quello che prende il nome di \textbf{\textit{profilo a doppia T}} (figura \ref{trave-sezionet}).
		
		\figura{4}{1.8}{trave-sezionet}{sezione di una trave con profilo a doppia $T$ (o anche denominata sezione ad $H$).}{trave-sezionet}
		
		Questo tipo di sezione presenta delle inerzie $\Ixx \gg \Iyy$, quindi è molto efficace a resistere a momenti $M_x$, mentre non è possibile affermare altrettanto per momenti $M_y$: per migliorare la risposta del materiale ad entrambe le sollecitazioni (pur mantenendo la stessa area superficiale) è possibile utilizzare dei profilati a sezione cava, come in figura \ref{trave-profilati}.
		\figura{6}{1}{trave-profilati}{esempi di profilati con sezioni cave.}{trave-profilati}
		
		Il vantaggio di queste sezioni è che, nonostante la risposta alla flessione non sia ottimale come nel caso dei profili a doppia T, questo tipo di sezioni sono ottimali per resistere alla torsione.
		
	\subsection{Estensione dei risultati a travi più generiche}
		
		Come nell'analisi della flessione normale è possibile estendere i risultati ottenuti per:
		\begin{itemize}
			\item sezioni ad asse non rettilineo ma per il quale il raggio di curvatura $R$ è molto maggiore della dimensione caratteristica $d$ della sezione; in particolare per $\frac dR = \frac 1 20$ (quindi per un rapporto neanche \textit{troppo elevato}) l'errore che si commette nel valutare lo stato di tensione $\szz$ è di circa $1.7\%$ e dunque il modello può essere ritenuto valido. Per travi con curvatura elevata è opssibile utilizzare delle teorie correttive rispetto all'approccio di Navier: in questo caso si osserva che l'asse neutro non è più baricentrico;
			
			\item se la sezione della trave varia blandamente lungo l'asse, la trave può essere considerata valida (se la generatrice del cono associato alla variazione di asse presenta un angolo di $5^\circ$, l'errore che si commette nel valutare lo stato di tensione è inferiore al $0.15\%$);
			
			\item nel caso di momenti flettenti concentrati anche in questo caso è possibile utilizzare il modello di Saint Venant introducendo delle congrue zone di estinzione;
			
			\item se le sezioni variano bruscamente è necessario introdurre delle zone di estinzione e utilizzare dei fattori di forma $k_t$ come descritti a pagina \pageref{conc:sv:fattoreforma} (concetto \ref{conc:sv:fattoreforma});
		\end{itemize}
	
	\subsection{Analisi della flessione deviata}
		In generale una trave è sottoposta ad un momento $\ov M = M_x \vers i + M_y\vers j$ di orientazione non necessariamente principale. Definito $M$ il modulo del momento $\ov M$ e scelto $\gamma$ l'angolo che tale vettore genera con l'asse $x$, allora le componenti possono essere espresse come $M_x = M \cos\gamma$ e $M_y = M\sin\gamma$.
		
		\begin{concetto}
			Utilizzando il principio di sovrapposizione degli effetti è possibile determinare lo stato di tensione $\szz$ dovuto alla composizione degli stati dovuti ad $M_x$ e $M_y$; è dunque matematicamente possibile determinare l'equazione dell'\textbf{asse neutro} di una trave soggetto a flessione deviata tramite l'espressione
			\begin{equation} \label{eq:sv:asseneutrodeviata}
				y = \frac \My \Mx \frac \Ixx \Iyy x = \underbrace{\tan\gamma\, \frac \Ixx \Iyy}_{\tan\beta} x
			\end{equation}
			In questa espressione $\beta$ rappresenta dunque l'inclinazione dell'asse neutro della trave soggetta a flessione deviata.
		\end{concetto}
		\begin{dimostrazione}
			L'equazione \ref{eq:sv:asseneutrodeviata} può essere dimostrata considerando che l'asse neutro è quello per il quale $\szz = 0$; l'equazione da dimostrare è ottenuta per inversione dell'espressione della tensione ottenuta considerando il principio di sovrapposizione degli effetti:
			\[ \szz = \sigma_{zz,M_x} + \sigma_{zz, M_y} = \frac \Mx \Ixx y - \frac \My \Iyy x = 0\]
		\end{dimostrazione}
		\begin{osservazione}
			Nel caso di flessione retta è possibile osservare che l'asse neutro e l'asse del momento coincidono sempre, mentre questo non vale in generale per la flessione deviata! In particolare si osserva che se $\Ixx > \Iyy$ allora $\beta > \gamma$, se $\Ixx < \Iyy$ allora $\beta < \gamma$, mentre è solo nella situazione in cui $\Ixx = \Iyy$ che $\beta = \gamma$  (come nel caso della flessione retta).
		\end{osservazione}
	
		Determinata dunque la direzione $\vers n$ dell'asse neutro, allora l'andamento di $\szz$ sarà lineare a partire dall'asse neutro, come è possibile osservare in figura \ref{tens-fless-dev}.
	
		\figura{5}{1.5}{tens-fless-dev}{valore della tensione $\szz$ in una trave soggetta a momento flettente di asse neutro $\vers n$.}{tens-fless-dev}
		
		Il \textbf{campo di spostamento} dovuto alla flessione retta si potrebbe ricavare per integrazione delle equazioni di congruenza, tuttavia risulta più immediato applicare il principio di sovrapposizione degli effetti per determinare le componenti del campo $\ov u$:
		\begin{equation} \label{eq:sv:campodeviato}
		\ov u = \left\{ 
		\begin{split}
		u &= \frac \My{2E\Iyy}z^2 + \nu \left[ -\frac{\Mx}{E\Ixx}xy + \frac{\My}{E\Iyy}\big(x^2-y^2\big) \right]\\
		v &= -\frac \Mx{2E\Ixx}z^2 + \nu \left[ \frac{\Mx}{E\Ixx}\big(x^2-y^2\big) + \frac{\My}{E\Iyy}xy \right] \\
		w &= \left( \frac \Mx{E\Ixx} y - \frac{\My}{E\Iyy}x \right)z
		\end{split}\right.
		\end{equation}
		Anche in questo caso è possibile osservare che $\ov u$ presenta le \textbf{componenti primarie} (indipendenti da $\nu$) e le \textbf{componenti secondarie} (dipendenti da $\nu$).
		
		\begin{concetto}
			Il rapporto tra le componenti del campo primario $u$ e $v$ (eq. \ref{eq:sv:campodeviato}) permette di determinare l'inclinazione $\beta'$ dell'\textbf{asse di flessione} che risulta valere
			\begin{equation}
				\frac u v = -\frac \Mx \My \frac \Iyy \Ixx = \tan\beta'
			\end{equation}
		\end{concetto}
		\begin{osservazione}
			Considerando i risultati sull'asse neutro (la cui inclinazione $\beta$ è descritta dall'equazione \ref{eq:sv:asseneutrodeviata}), è possibile verificare che asse neutro e di flessione sono mutuamente ortogonali, in quanto
			\[ \tan \beta \, \tan \beta' = -1 \qquad \Leftrightarrow \qquad \beta \perp \beta'  \]
		\end{osservazione}
		L'asse deformato, di inclinazione $\beta'$, giace dunque nel piano perpendicolare all'asse neutro (in quanto esso ha inclinazione $\beta\perp\beta'$); scelto come sistema di riferimento quello determinato da questi due assi è possibile valutare lo spostamento $\delta$ dei punti dall'asse neutro e l'inclinazione $d\delta/dz$ dell'asse lungo la trave come
		\[ \delta = \sqrt{u^2+v^2} = \frac {z^2}{2E} \sqrt{ \left( \frac {\Mx}\Ixx\right)^2 + \left( \frac {M_y}\Iyy\right)^2}\qquad \frac{d\delta}{dz} = \frac z {2E}\sqrt{ \left( \frac {M_x}\Ixx\right)^2 + \left( \frac {M_y}\Iyy\right)^2 } \]
		E' anche possibile definire la curvatura $k_f$ nel piano di flessione come
		\[ k_f = \frac 1 E \sqrt{ \left( \frac {\Mx}\Ixx\right)^2 + \left( \frac {M_y}\Iyy\right)^2} \]
		
		\begin{concetto}
			Definiti $m$ l'\textbf{asse del momento} (di angolo $\gamma$), $s$ l'\textbf{asse di sollecitazione} (angolo $\gamma'$), $n$ l'\textbf{asse neutro} (di inclinazione $\beta$) e $f$ l'\textbf{asse di flessione} (angolo $\beta'$) è possibile affermare che $m\perp s$ e $n\perp f$ e l'inclinazione $\beta$ si ricava da $\gamma$ tramite le relazioni
			\[ \tan\beta = \tan\gamma \frac \Ixx \Iyy \qquad \leftrightarrow\qquad \tan\beta \tan\gamma' = - \frac \Ixx \Iyy \]
			Da queste relazioni è possibile dunque affermare che che asse di sollecitazione e asse neutro sono coniugati rispetto all'ellissoide centrale di inerzia e può dunque essere ricavato per via grafica:
			\begin{center}
				\includegraphics[width=5cm]{sv-assiconiugati}
			\end{center}			
		\end{concetto}
		Dato dunque l'asse dei momenti $m$ si determina l'asse di sollecitazione $s$ ad esso ortogonale: l'asse neutro $n$, passante per l'origine, avrà dunque direzione parallela alle rette tangenti l'ellissoide di inerzia nei punti di intersezione con $s$.
		
	\subsection{Analisi dell'azione normale eccentrica}
		Nel caso di trave sottoposta ad azioni interne di componenti $N, \Mx,\My$ allora l'asse neutro non è più baricentrico (per via dell'azione $N$); invertendo il campo di tensione $\szz$ in questo caso (eq. \ref{eq:sv:szznormale}, pag. \pageref{eq:sv:szznormale}) è possibile determinare l'equazione dell'asse neutro imponendo che il campo scalare deve essere nullo, ottenendo la relazione
		\[ y = x\, \tan\beta + y_0 \qquad \leftarrow\quad \tan\beta= \frac \My \Mx \frac \Ixx \Iyy \quad y_0 = - \frac N A \frac \Ixx \Mx  \]
		
	    Avendo dunque la presenza simultanea dell'azione normale $N$ e dei momenti flettenti $M_x,M_y$, tutti applicati nel centro del baricentro della trave (coordinate $x=y=0$), l'idea è quella di considerare una sola azione normale eccentrica $N$ in un punto preciso in modo da ottenere un sistema staticamente equivalente. Definito tale punto $C$ di coordinate $(x_C, y_C)$, allora per equivalenza statica è possibile determinare il punto:
	    \begin{align*}
	    	y_C \ &| \ N y_C = M_x \qquad  & \Rightarrow \quad y_C &= \frac \Mx N \\
	    	x_C \ &| \ N x_C = -M_y \qquad & \Rightarrow \quad x_C &= - \frac \My N 
	    \end{align*}
    	A questo punto è possibile riscrivere l'equazione dello stato di tensione $\szz$ nel caso di azione normale eccentrica utilizzando le espressioni appena determinate per esprimere i momenti in funzione del centro di applicazione $C$; in particolare considerando le relazioni che legano i momenti di inerzia con i relativi raggi, è possibile verificare le seguenti relazioni per quanto riguarda lo stato di tensione $\szz$:
    	\begin{equation} \label{eq:sv:tensioneeccentrica}
	    	\szz = \frac N A + \frac {Ny_C}\Ixx y + \frac{Nx_C}{\Iyy}x = \frac N A \left( 1 + \frac{y_C}{\rho_x^2}y + \frac{x_C}{\rho_y^2}x \right)
    	\end{equation}
    	
    	Equivalentemente è possibile riscrivere l'energia potenziale elastica $U_e$ considerando la posizione eccentrica di $N$ in modo che la stessa possa considerare i contributi di momenti $\Mx$ ed $\My$:
    	\[ U_e = \frac{N^2}{2EA}L \left( 1 + \frac{y_C^2}{\rho_x^2} + \frac{x_C^2}{\rho_y^2} \right)  \]
    	
    	\paragraph{Asse neutro} A questo punto per determinare l'asse neutro, ossia la retta per la quale $\szz = 0$, è sufficiente porre uguale a zero il termine tra parentesi in equazione \ref{eq:sv:tensioneeccentrica}:
    	\[ 1 +\frac{y_C}{\rho_x^2} y + \frac{x_C}{\rho_y^2}x = 0 \qquad \Rightarrow \quad \frac{y}{-\rho_x^2/y_C} + \frac x {-\rho_y^2/x_C} = \frac y {y_0} + \frac{x}{x_0} = 1  \]
    	Di questa relazione è possibile verificare che $x_0,y_0$ rappresentano i valori delle intercette dell'asse neutro con i rispettivi assi cardinali e che deve verificarsi $y_0 y_C = -\rho_x^2 $ e $x_0x_C = \rho_y^2$. 
    	\begin{osservazione}
    		Dovendo essere le grandezze $\rho_{x,y}^2$ strettamente positive, allora è possibile osservare che il punto di coordinate $\big(x_0,y_0\big)$ si trova nel quadrante opposto al centro di sollecitazione $C = \big(x_C,y_C\big)$.
    	\end{osservazione}
    	
    	\begin{osservazione}
    		Tanto più vicino il punto $C$ è all'origine, tanto più lontano risulterà essere l'asse neutro: infatti se il punto tende a sovrapporsi al baricentro ci si trova in una situazione in cui $\Mx,\My \approx 0$ e l'unica azione presente è quella dovuta all'azione $N$ della quale si può pensare che l'asse neutro è posto ad una distanza infinita dal corpo stesso.
    	\end{osservazione}
    	
    	\begin{concetto}
    		Si definisce \textbf{nocciolo d'inerzia} il luogo dei punti $C$ di applicazione del carico eccentrico che producono assi neutri che sono esterni alla struttura.
    	\end{concetto}
    	Il nocciolo d'inerzia è importante in quanto permette di stabilire per quali punti di applicazione del carico eccentrico la struttura si trova con uno stato di tensione di sola compressione/trazione.
    	
\section{Analisi della torsione}
	Come visto a pagina \pageref{eq:sv:equivstatica} ad un \textbf{momento torcente} $M_z$ è associato non più il campo di tensione scalare $\szz$, ma bensì il campo vettoriale di tensioni tangenziali $\ov \tau = \tyz \vers i + \txz \vers j$.
	
	Studiare il campo $\tau$ analiticamente risulta essere complesso in maniera generale, per questo la soluzione esatta si ricava per geometrie relativamente semplice, mentre soluzioni approssimate possono essere ricavate per sezioni a forma compatta o a parete sottile.
	
	\subsection{Sezione circolare}
		La risposta alla torsione di una trave a sezione circolare venne inizialmente proposta da Coulomb e si basa sull'osservazione sperimentale che che le sezioni della trave stessa, dopo deformazione, risultano essere inalterate (in forma) seppur variando l'orientazione relativa rispetto all'asse del cilindro stesso.
		
		Avendo ipotizzato dunque che lo spostamento $\ov u_P$ di un punto $P$ descritto dal vettore $\ov r_P$ della trave sia rigido, allora è possibile definire lo stesso tramite la relazione $\ov u_P = \cancel{\ov u_G} + \ov \theta_z \times \ov r_p$ (essendo per ipotesi il sistema baricentrico e sapendo che lo stesso non si muove, allora $\ov u_G = 0$) e dunque si può esprimere come $\ov u_P = \ov \theta_z \times \ov r_P$ con $\ov \theta_z = \theta_z(z) \vers k$. Espandendo il prodotto vettoriale è possibile calcolare in maniera estesa il campo di spostamento del punto $P$ di coordinate $(x,y,z)$ come
		\[ \ov u_P = \begin{pmatrix}
			-\theta_z y \\ \theta_z x \\ 0
		\end{pmatrix} \]
    	\begin{nota}
    	    Per via del prodotto scalare è possibile osservare che il vettore di spostamento $\ov u_P$ è perpendicolare alla posizione originaria $\ov r_P$ del punto $P$ stesso.
    	\end{nota}
    	
    	Considerando che il momento torcente $M_z$ applicato sia costante lungo tutta la trave, allora ogni concio elementare che la costistuisce è deformato in maniera eguale, ossia ogni elemento infinitesimo risulterà essere relativamente ruotato di uno stesso angolo $d\theta_z$.
    	
    	\begin{concetto}
    	    Nell'ipotesi di momento torcente $M_z$ applicato costante allora la distorsione angolare delle sezioni elementari è costante e si definisce \textbf{gradiente di torsione} $\Theta$ la grandezza
    	    \begin{equation} \label{eq:sv:gradientetorsione}
    	        \Theta =\frac{d\theta_z}{dz} =\textrm{costante}
    	    \end{equation}
    	\end{concetto}
    	Con questa relazione è possibile calcolare che la rotazione relatica $\theta_z$ tra due sezioni poste ad una distanza $l$ vale $\theta_z(l)=\Theta l$.
    	\begin{concetto}
    	    Sfruttando il risultato dell'equazione \ref{eq:sv:gradientetorsione} è possibile calcolare in maniera esplicita il \textbf{campo di spostamento} $\ov u$ della trave deformata come
    	    \begin{equation}
    	        \ov u = \begin{cases}
    	            u =- \Theta z y \\ v =\Theta z x \\ w =0
    	        \end{cases}
    	    \end{equation}
    	\end{concetto}
    	Si dimostra che questa deformazione non comporta elongazione specifica ($\exx = \eyy = \ezz = 0$) mentre si ha l'introduzione di distorsioni angolari che possono essere ricavate dalle equazioni di congruenza:
    	\[ \gxz = \pd u z + \pd w x = -\Theta y \qquad \gyz = \pd v z + \pd w u =\Theta x \qquad \gxy = \pd u y + \pd v x = 0 \]
    	Dall'analisi delle distorsioni angolari è possibile stabilire le componenti dello \textbf{stato di tensione} che risultano valere $\txz = G \gxz = - G \Theta y$ e $\tyz =G \gyz =G \Theta y$.
    	
    	Con questo stato di tensione è possibile verificare che anche le condizioni al contorno sono rispettate, in quanto in ogni punto della circonferenza presenta uno stato di tensione tangente al mantello.
    	
    	\begin{dimostrazione}
    	    Scelto un punto $P$ sul mantello di posizione $\ov r_P=\big( x_P, y_P\big)$, allora il versore $\vers n_\Gamma$ normale al mantello può essere ottenuto dalla normalizzazione dello stesso:
    	    \[ \vers n_\Gamma = \left( \frac{x_P}{r_P} \ , \ \frac{y_P}{r_P} \right)  \]
    	    Noto che lo stato di tensione nel punto vale $\ov \tau = \big( - G\Theta y_P, G\Theta x_P\big)$ allora effettuando il prodotto scalare $\vers n_\Gamma \cdot \ov \tau$ si osserva che esso risulta essere sempre nullo, e dunque $\ov \tau$ è perpendicolare a $\vers n_\Gamma$ (e conseguentemente tangenziale al mantello).
    	\end{dimostrazione}
    	In generale è anche possibile calcolare il modulo del campo $\ov \tau$ utilizzando il teorema di Pitagora, dimostrando che esso dipende direttamente dalla distanza $r$ del punto rispetto al quale si valuta lo stato di tensione rispetto all'origine secondo l'equazione:
    	\[ \|\ov \tau \| = \sqrt{\txz ^2 + \tyz^2} =G\Theta \sqrt{x^2+y^2}=G\Theta r \]
    	Da questa relazione si ricava che il valore massimo di tensione lo si trova in corrispondenza del mantello (punto che massimizza la distanza $r$ dall'origine della terna di riferimento).
    	
    	\paragraph{Calcolo del gradiente di torsione} Per calcolare il gradiente di torsione $\Theta$ causato dal momento torcente $M_z$, dipendente dalla geometria della sezione, è sufficiente considerare le equivalenze statiche che gli stati di tensione generano; in particolare essendo il sistema baricentrico, il campo vettoriale $\ov \tau$ genera delle azioni interne $T_x, T_y$ sempre nulle.
    	\begin{dimostrazione}
    		Per equivalenza statica si dimostra infatti che
    		\[ T_x = \int_A \txz \, dA = -G\Theta\int_A y\, dA = - G\Theta S_x \xrightarrow{S_x = 0} 0 \]
    		Questo vale in quanto il sistema cartesiano di riferimento è baricentrico; dimostrazione analoga si può effettuare per calcolare $T_y$ che risulta valere $G\Theta S_y = 0$.
    	\end{dimostrazione}
    
    	A questo punto impostando l'equivalenza statica per il momento torcente si osserva che esso può essere definito a partire del \textbf{momento polare} $I_p$:
    	\[ M_z = \int_A \Big( \tyz x - \txz y\Big)\, dA = G\Theta \int_A\big(x^2 + y^2\big)\, dA = G\Theta \big(\Ixx + \Iyy\big) = G\Theta I_p  \]
    	
    	\begin{concetto}
    		Invertendo l'equazione precedente è possibile stabilire il \textbf{gradiente di torsione} di una trave a sezione circolare sottoposta a momento torcente costante tramite la relazione
    		\begin{equation}
    			\Theta = \frac{d\theta_z}{dz}= \frac{M_z}{G I_p}
    		\end{equation}
    		Sfruttando questa relazione è possibile riscrivere le componenti scalari della tensione $\ov \tau$ e il campo di spostamento secondo le relazioni
    		\begin{equation}
    			\txz = -G\Theta y = - \frac{M_z}{I_p}y \qquad \tyz = G\Theta x = \frac{M_z}{I_p}x \qquad
    			\ov u = \begin{cases}
    				u =- \frac{M_z}{GI_p} z y \\ v =\frac{M_z}{GI_p} z x \\ w =0
    			\end{cases}
    		\end{equation}
    		E' inoltre possibile calcolare l'\textbf{energia potenziale elastica specifica} $dU_e /dz$ lungo l'asse della trave tramite integrazione del potenziale elastico $\phi = \psi = \tau^2/2G$:
    		\begin{equation}
    			\frac{dU_e}{dz} = \int_A\psi \, dA = \frac{M_z^2}{2GI_p^2} \int_Ar^2\, dA  = \frac{M_z^2}{2GI_p}
    		\end{equation}
    	\end{concetto}
    	
    	\paragraph{Modulo di resistenza a torsione} Noto il diametro $d$ della sezione, il valore del momento polare di inerzia vale $I_p = \frac \pi {32}d^4$
    	
    \subsection{Sezioni cave a parete sottile}
    	Considerando una tubazione a sezione circolare cava con diametro esterno $d_e$ ed interno $d_i$ con spessore sottile $s$, se lo stesso è molto piccolo è possibile approssimare la sezione ad una circonferenza con diametro $d$ pari al valor medio della sezione stessa:
    	\[ d_e \simeq d_i \simeq d = \frac{d_e - d_i}{2}\]
    	
    	L'equazione della sezione circolare piena prevede che il valore massimo di tensione tangenziale $\tau_{max}$ sia misurato in corrispondenza del mantello ad una distanza $r = d/2$, secondo la relazione $\frac{M_z}{I_p}\frac d 2$; noto che il momento polare di inerzia $I_p$ della corona circolare vale $\frac{\pi}{32}$
    	
    	
    	
    	
    	
    	
    	
    	
    	
    	
    	
    	
    	
    	
    	
    	
    	
    	
    	
    	
    	
    	
    	
    	
    