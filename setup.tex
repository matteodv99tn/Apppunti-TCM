\usepackage[width=15.00cm, height=23.00cm]{geometry}
\usepackage[italian]{babel}
\usepackage[utf8]{inputenc}
\usepackage[T1]{fontenc}
\usepackage{dsfont}
\usepackage{amsmath}
\usepackage{amsfonts}
\usepackage{amssymb}
\usepackage{graphicx}
\usepackage{paracol}
\usepackage{xparse}
\usepackage{sidecap}
\usepackage[makeroom]{cancel}
\usepackage{capt-of}
\usepackage{caption}
\usepackage[dvipsnames]{xcolor}
\usepackage{xpatch}
\usepackage{subcaption}
\usepackage[most]{tcolorbox}
\usepackage{lipsum}
\usepackage{float}
\usepackage{imakeidx}
\usepackage{wrapfig}
\usepackage{marginnote}
\usepackage{ upgreek }
\usepackage{bm}
\usepackage{enumerate}

\graphicspath{{Immagini/}}

%\newcommand{\de}[1]{\textbf{\textcolor{NavyBlue}{#1}}}

\makeindex[columns=3, title=Indice Analitico, intoc]

\captionsetup{font = {it, small}, labelfont={color=NavyBlue, bf}}


\renewcommand{\H}{\mathcal H}
\renewcommand{\S}{\mathcal S}
\newcommand{\F}{\mathcal F}
\newcommand{\E}{\mathcal E}
\newcommand{\R}{\mathcal R}
\newcommand{\I}{\mathcal I}
\renewcommand{\L}{\mathcal L}
\newcommand{\C}{\mathcal C}
\newcommand{\D}{\mathcal D}


\newcommand{\ov}[1]{ \vett{#1}  	}
\newcommand{\vers}[1]{ \boldsymbol{\hat{#1}}  	}
\newcommand{\vett}[1]{ \boldsymbol{#1}  	}
\newcommand{\dov}[1]{ \dot{\boldsymbol{#1}  }	}
\newcommand{\pd}[2]{\frac{\partial #1}{\partial #2}}
\newcommand{\inv}{\textrm{I}}
\newcommand{\exx}{\epsilon_{xx}}
\newcommand{\exy}{\epsilon_{xy}}
\newcommand{\eyy}{\epsilon_{yy}}
\newcommand{\ezz}{\epsilon_{zz}}
\newcommand{\txy}{\tau_{xy}}
\newcommand{\txz}{\tau_{xz}}
\newcommand{\tyz}{\tau_{yz}}
\newcommand{\sxx}{\sigma_{xx}}
\newcommand{\syy}{\sigma_{yy}}
\newcommand{\szz}{\sigma_{zz}}


\newcommand{\Mx}{M_x}
\newcommand{\My}{M_y}
\newcommand{\Mz}{M_z}
\newcommand{\Ixx}{I_{xx}}
\newcommand{\Iyy}{I_{yy}}
\newcommand{\Izz}{I_{zz}}

\newcommand{\opdiff}{\Big[\ \partial \ \Big] }
\newcommand{\matalpha}{\Big[\ \alpha \ \Big] } 
\newcommand{\asmat}[1]{ \big[ \ #1 \ \big] }

\newcommand{\figura}[5]{\begin{SCfigure}[#2][b!h!t!]
		\centering
		\includegraphics[width=#1 cm]{#3}
		\caption{#4} \label{#5}
\end{SCfigure}}
















\newcommand{\bfcolor}[1]{\renewcommand*{\textbf}[1]{{\bfseries { \color{#1} ##1 }}}}

\setcolumnwidth{0.3\textwidth}



\newcounter{concetti}
\newenvironment{concetto}{
	
	\bfcolor{NavyBlue}
	\refstepcounter{concetti}
	{\color{NavyBlue}\textbf{Concetto \theconcetti:}} \quad
}{
}
\numberwithin{concetti}{chapter}
\tcolorboxenvironment{concetto}{
	boxrule=0pt,
	boxsep=0pt,
	colback={White!90!NavyBlue},
	enhanced jigsaw, 
	borderline west={2pt}{0pt}{NavyBlue},
	sharp corners,
	before skip=5pt,
	after skip=10pt,
	breakable,
}

\newcounter{teoremi}
\newenvironment{teorema}[2]{
	\bfcolor{ForestGreen}
	\refstepcounter{teoremi}
	\textbf{Concetto \theteoremi #1} 
	\vspace{3mm} 
	
	\texttt{Ipotesi: } #2
	
	\vspace{3mm} 
	
	\texttt{Enunciato}: 
}{
}
\numberwithin{teoremi}{chapter}
\tcolorboxenvironment{teorema}{
	boxrule=0pt,
	boxsep=0pt,
	colback={White!100!ForestGreen},
	enhanced jigsaw, 
	borderline west={2pt}{0pt}{ForestGreen},
	sharp corners,
	before skip=5pt,
	after skip=10pt,
	breakable,
}


\newenvironment{dimostrazione}{
	\bfcolor{Orchid}
	\textbf{Dimostrazione} \quad
}{ }
\tcolorboxenvironment{dimostrazione}{
	boxrule=0pt,
	boxsep=0pt,
	colback={White!100!NavyBlue},
	enhanced jigsaw, 
	borderline west={2pt}{0pt}{Orchid},
	sharp corners,
	before skip=5pt,
	after skip=10pt,
	breakable,
}

\newenvironment{osservazione}{
	\bfcolor{BurntOrange}
	\textbf{Osservazione: }
}{ }
\tcolorboxenvironment{osservazione}{
	boxrule=0pt,
	boxsep=0pt,
	colback={White!100!BurntOrange},
	enhanced jigsaw, 
	borderline west={2pt}{0pt}{BurntOrange},
	sharp corners,
	before skip=5pt,
	after skip=10pt,
	breakable,
}

\newenvironment{nota}{
	\bfcolor{CadetBlue}
	\textbf{Nota: }
}{ }
\tcolorboxenvironment{nota}{
	boxrule=0pt,
	boxsep=0pt,
	colback={White!100!CadetBlue},
	enhanced jigsaw, 
	borderline west={2pt}{0pt}{CadetBlue},
	sharp corners,
	before skip=5pt,
	after skip=10pt,
	breakable,
}




\newcounter{numrichiamo}
\newenvironment{richiamo}{
	\noindent
	\refstepcounter{numrichiamo}
	
%	\renewcommand{\de}[1]{ { \color{ForestGreen} \textbf{#1} } }
	
	{\color{ForestGreen}\textbf{Richiamo \thenumrichiamo:}}
}{
}
\tcolorboxenvironment{richiamo}{
	boxrule=0pt,
	boxsep=0pt,
	colback={White!90!ForestGreen},
	enhanced jigsaw, 
	borderline west={2pt}{0pt}{ForestGreen},
	sharp corners,
	before skip=5pt,
	after skip=5pt,
	breakable,
}

\newcounter{esempi}
\numberwithin{esempi}{chapter}
\newenvironment{esempio}[1]{
	\noindent
	\refstepcounter{esempi}
	{\color{Periwinkle}\textbf{Esempio \theesempi#1} \\ } 
	
	
	\noindent
	\renewcommand{\de}[1]{\textbf{\textcolor{Periwinkle}{#1}}}
}{
}


\tcolorboxenvironment{esempio}{
	boxrule=0pt,
	boxsep=0pt,
	colback={White!90!Periwinkle},
	enhanced jigsaw, 
	borderline west={2pt}{0pt}{Periwinkle},
	sharp corners,
	before skip=10pt,
	after skip=10pt,
	breakable,
}
